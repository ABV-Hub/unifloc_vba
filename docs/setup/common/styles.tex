%%% Шаблон %%%
\DeclareRobustCommand{\todo}{\textcolor{red}}       % решаем проблему превращения названия цвета в результате \MakeUppercase, http://tex.stackexchange.com/a/187930, \DeclareRobustCommand protects \todo from expanding inside \MakeUppercase
\AtBeginDocument{%
    \setlength{\parindent}{2.5em}                   % Абзацный отступ. Должен быть одинаковым по всему тексту и равен пяти знакам (ГОСТ Р 7.0.11-2011, 5.3.7).
}

%%% Подписи %%%
\setlength{\abovecaptionskip}{0pt}   % Отбивка над подписью
\setlength{\belowcaptionskip}{0pt}   % Отбивка под подписью
\captionwidth{\linewidth}
\normalcaptionwidth

%%% Таблицы %%%
\ifnumequal{\value{tabcap}}{0}{%
    \newcommand{\tabcapalign}{\raggedright}  % по левому краю страницы или аналога parbox
    \renewcommand{\tablabelsep}{~\cyrdash\ } % тире как разделитель идентификатора с номером от наименования
    \newcommand{\tabtitalign}{}
}{%
    \ifnumequal{\value{tablaba}}{0}{%
        \newcommand{\tabcapalign}{\raggedright}  % по левому краю страницы или аналога parbox
    }{}

    \ifnumequal{\value{tablaba}}{1}{%
        \newcommand{\tabcapalign}{\centering}    % по центру страницы или аналога parbox
    }{}

    \ifnumequal{\value{tablaba}}{2}{%
        \newcommand{\tabcapalign}{\raggedleft}   % по правому краю страницы или аналога parbox
    }{}

    \ifnumequal{\value{tabtita}}{0}{%
        \newcommand{\tabtitalign}{\par\raggedright}  % по левому краю страницы или аналога parbox
    }{}

    \ifnumequal{\value{tabtita}}{1}{%
        \newcommand{\tabtitalign}{\par\centering}    % по центру страницы или аналога parbox
    }{}

    \ifnumequal{\value{tabtita}}{2}{%
        \newcommand{\tabtitalign}{\par\raggedleft}   % по правому краю страницы или аналога parbox
    }{}
}

\precaption{\tabcapalign} % всегда идет перед подписью или \legend
\captionnamefont{\normalfont\normalsize} % Шрифт надписи «Таблица #»; также определяет шрифт у \legend
\captiondelim{\tablabelsep} % разделитель идентификатора с номером от наименования
\captionstyle[\tabtitalign]{\tabtitalign}
\captiontitlefont{\normalfont\normalsize} % Шрифт с текстом подписи

%%% Рисунки %%%
\setfloatadjustment{figure}{%
    \setlength{\abovecaptionskip}{0pt}   % Отбивка над подписью
    \setlength{\belowcaptionskip}{0pt}   % Отбивка под подписью
    \precaption{} % всегда идет перед подписью или \legend
    \captionnamefont{\normalfont\normalsize} % Шрифт надписи «Рисунок #»; также определяет шрифт у \legend
    \captiondelim{\figlabelsep} % разделитель идентификатора с номером от наименования
    \captionstyle[\centering]{\centering} % Центрирование подписей, заданных командой \caption и \legend
    \captiontitlefont{\normalfont\normalsize} % Шрифт с текстом подписи
    \postcaption{} % всегда идет после подписи или \legend, и с новой строки
}

%%% Подписи подрисунков %%%
\newsubfloat{figure} % Включает возможность использовать подрисунки у окружений figure
\renewcommand{\thesubfigure}{\asbuk{subfigure}}           % Буквенные номера подрисунков
\subcaptionsize{\normalsize} % Шрифт подписи названий подрисунков (не отличается от основного)
\subcaptionlabelfont{\normalfont}
\subcaptionfont{\!\!) \normalfont} % Вот так тут добавили скобку после буквы.
\subcaptionstyle{\centering}
%\subcaptionsize{\fontsize{12pt}{13pt}\selectfont} % объявляем шрифт 12pt для использования в подписях, тут же надо интерлиньяж объявлять, если не наследуется

%%% Настройки гиперссылок %%%
\ifluatex
    \hypersetup{
        unicode,                % Unicode encoded PDF strings
    }
\fi

\hypersetup{
    linktocpage=true,           % ссылки с номера страницы в оглавлении, списке таблиц и списке рисунков
%    linktoc=all,                % both the section and page part are links
%    pdfpagelabels=false,        % set PDF page labels (true|false)
    plainpages=false,           % Forces page anchors to be named by the Arabic form  of the page number, rather than the formatted form
    colorlinks,                 % ссылки отображаются раскрашенным текстом, а не раскрашенным прямоугольником, вокруг текста
    linkcolor={linkcolor},      % цвет ссылок типа ref, eqref и подобных
    citecolor={citecolor},      % цвет ссылок-цитат
    urlcolor={urlcolor},        % цвет гиперссылок
%    hidelinks,                  % Hide links (removing color and border)
%    pdftitle={\Title},    % Заголовок
%    pdfauthor={\Author},  % Автор
%    pdfsubject={\thesisSpecialtyNumber\ \thesisSpecialtyTitle},      % Тема
%    pdfcreator={Создатель},     % Создатель, Приложение
%    pdfproducer={Производитель},% Производитель, Производитель PDF
%    pdfkeywords={\keywords},    % Ключевые слова
    pdflang={ru},
}
\ifnumequal{\value{draft}}{1}{% Черновик
    \hypersetup{
        draft,
    }
}{}

%%% Списки %%%
% Используем короткое тире (endash) для ненумерованных списков (ГОСТ 2.105-95, пункт 4.1.7, требует дефиса, но так лучше смотрится)
\renewcommand{\labelitemi}{\normalfont\bfseries{--}}

% Перечисление строчными буквами латинского алфавита (ГОСТ 2.105-95, 4.1.7)
%\renewcommand{\theenumi}{\alph{enumi}}
%\renewcommand{\labelenumi}{\theenumi)} 

% Перечисление строчными буквами русского алфавита (ГОСТ 2.105-95, 4.1.7)
\makeatletter
\AddEnumerateCounter{\asbuk}{\russian@alph}{щ}      % Управляем списками/перечислениями через пакет enumitem, а он 'не знает' про asbuk, потому 'учим' его
\makeatother
%\renewcommand{\theenumi}{\asbuk{enumi}} %первый уровень нумерации
%\renewcommand{\labelenumi}{\theenumi)} %первый уровень нумерации 
\renewcommand{\theenumii}{\asbuk{enumii}} %второй уровень нумерации
\renewcommand{\labelenumii}{\theenumii)} %второй уровень нумерации 
\renewcommand{\theenumiii}{\arabic{enumiii}} %третий уровень нумерации
\renewcommand{\labelenumiii}{\theenumiii)} %третий уровень нумерации 

\setlist{nosep,%                                    % Единый стиль для всех списков (пакет enumitem), без дополнительных интервалов.
    labelindent=\parindent,leftmargin=*%            % Каждый пункт, подпункт и перечисление записывают с абзацного отступа (ГОСТ 2.105-95, 4.1.8)
}
