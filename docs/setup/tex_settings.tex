% файл с настройками документов, для того чтобы транслировать разные настройки по нескольким документам
% rnt 2018

% version definition
\newcommand{\unf}{Unifloc 7.12 VBA}

%% Режим черновика
\makeatletter
\@ifundefined{c@draft}{
  \newcounter{draft}
  \setcounter{draft}{0}  % 0 --- чистовик (максимальное соблюдение ГОСТ)
                         % 1 --- черновик (отклонения от ГОСТ, но быстрая сборка итоговых PDF)
}{}
\makeatother


%%% Использование в xelatex и lualatex семейств шрифтов %%%
\makeatletter
\@ifundefined{c@fontfamily}{
  \newcounter{fontfamily}
  \setcounter{fontfamily}{1}  % 0 --- CMU семейство. Используется как fallback;
                              % 1 --- Шрифты от MS (Times New Roman и компания)
                              % 2 --- Семейство Liberation
}{}
\makeatother

%% Библиография


%%% Предкомпиляция tikz рисунков для ускорения работы %%%
\makeatletter
\@ifundefined{c@imgprecompile}{
  \newcounter{imgprecompile}
  \setcounter{imgprecompile}{0}   % 0 --- без предкомпиляции;
                                  % 1 --- пользоваться предварительно скомпилированными pdf вместо генерации заново из tikz
                                  % рнт - без предкомпиляции не работает
}{}
\makeatother
            % общие настройки шаблона
%%% Проверка используемого TeX-движка %%%
\RequirePackage{ifxetex, ifluatex}
\newif\ifxetexorluatex   % определяем новый условный оператор (http://tex.stackexchange.com/a/47579)
\ifxetex
    \xetexorluatextrue
\else
    \ifluatex
        \xetexorluatextrue
    \else
        \xetexorluatexfalse
    \fi
\fi

\RequirePackage{etoolbox}[2015/08/02]               % Для продвинутой проверки разных условий

%%% Поля и разметка страницы %%%
\usepackage{pdflscape}                              % Для включения альбомных страниц
\usepackage{geometry}                               % Для последующего задания полей

%%% Математические пакеты %%%
\usepackage{amsthm,amsmath,amscd}   % Математические дополнения от AMS
\usepackage{amsfonts,amssymb}       % Математические дополнения от AMS
\usepackage{mathtools}              % Добавляет окружение multlined
\usepackage{unicode-math}           % использование Unicode-шрифтов для формул

%%%% Установки для размера шрифта 14 pt %%%%
%% Формирование переменных и констант для сравнения (один раз для всех подключаемых файлов)%%
%% должно располагаться до вызова пакета fontspec или polyglossia, потому что они сбивают его работу
\newlength{\curtextsize}
\newlength{\bigtextsize}
\setlength{\bigtextsize}{13.9pt}

\makeatletter
%\show\f@size                                       % неплохо для отслеживания, но вызывает стопорение процесса, если документ компилируется без команды  -interaction=nonstopmode 
\setlength{\curtextsize}{\f@size pt}
\makeatother

%%% Кодировки и шрифты %%%

\usepackage{polyglossia}[2014/05/21]            % Поддержка многоязычности (fontspec подгружается автоматически)


%%% Оформление абзацев %%%
\usepackage{indentfirst}                            % Красная строка

%%% Цвета %%%
\usepackage[dvipsnames, table, hyperref, cmyk]{xcolor} % Совместимо с tikz. Конвертация всех цветов в cmyk заложена как удовлетворение возможного требования типографий. Возможно конвертирование и в rgb.

%%% Таблицы %%%
\usepackage{longtable,ltcaption}                    % Длинные таблицы
\usepackage{multirow,makecell}                      % Улучшенное форматирование таблиц
\usepackage{pbox}

%%% Общее форматирование
\usepackage{soulutf8}                               % Поддержка переносоустойчивых подчёркиваний и зачёркиваний
\usepackage{icomma}                                 % Запятая в десятичных дробях

%%% Оптимизация расстановки переносов и длины последней строки абзаца
\ifluatex
    \ifnumequal{\value{draft}}{1}{% Черновик
        \usepackage[hyphenation, lastparline, nosingleletter, homeoarchy,
        rivers, draft]{impnattypo}
    }{% Чистовик
        \usepackage[hyphenation, lastparline, nosingleletter]{impnattypo}
    }
\else
    \usepackage[hyphenation, lastparline]{impnattypo}
\fi

%%% Гиперссылки %%%
\usepackage{hyperref}[2012/11/06]

%%% Изображения %%%
\usepackage{graphicx}[2014/04/25]                   % Подключаем пакет работы с графикой

%%% Списки %%%
\usepackage{enumitem}

%%% Счётчики %%%
\usepackage[figure,table]{totalcount}               % Счётчик рисунков и таблиц
\usepackage{totcount}                               % Пакет создания счётчиков на основе последнего номера подсчитываемого элемента (может требовать дважды компилировать документ)
\usepackage{totpages}                               % Счётчик страниц, совместимый с hyperref (ссылается на номер последней страницы). Желательно ставить последним пакетом в преамбуле

%%% Продвинутое управление групповыми ссылками (пока только формулами) %%%
\usepackage{cleveref}                              % cleveref корректно считывает язык из настроек polyglossia

\creflabelformat{equation}{#2#1#3}                  % Формат по умолчанию ставил круглые скобки вокруг каждого номера ссылки, теперь просто номера ссылок без какого-либо дополнительного оформления
\crefrangelabelformat{equation}{#3#1#4\cyrdash#5#2#6}   % Интервалы в русском языке принято делать через тире, если иное не оговорено


\ifnumequal{\value{draft}}{1}{% Черновик
    \usepackage[firstpage]{draftwatermark}
    \SetWatermarkText{DRAFT}
    \SetWatermarkFontSize{14pt}
    \SetWatermarkScale{15}
    \SetWatermarkAngle{45}
}{}

%%% Цитата, не приводимая в автореферате:
% возможно, актуальна только для biblatex
%\newcommand{\citeinsynopsis}[1]{\ifsynopsis\else ~\cite{#1} \fi}


%%% Прикладные пакеты %%% 
%\usepackage{calc}               % Пакет для расчётов параметров, например длины

%%% Для добавления Стр. над номерами страниц в оглавлении
%%% http://tex.stackexchange.com/a/306950
\usepackage{afterpage}

\usepackage{tikz}                   % Продвинутый пакет векторной графики
\usetikzlibrary{chains}             % Для примера tikz рисунка
\usetikzlibrary{shapes.geometric}   % Для примера tikz рисунка
\usetikzlibrary{shapes.symbols}     % Для примера tikz рисунка
\usetikzlibrary{arrows}             % Для примера tikz рисунка
\ifnumequal{\value{imgprecompile}}{1}{% Только если у нас включена предкомпиляция
	\usetikzlibrary{external}   % подключение возможности предкомпиляции
	\tikzexternalize[prefix=Dissertation/images/] % activate! % здесь можно указать отдельную папку для скомпилированных файлов
	\ifxetex
	\tikzset{external/up to date check={diff}}
	\fi
}{}


\usepackage{tabu, tabulary}  %таблицы с автоматически подбирающейся шириной столбцов
\usepackage{fr-longtable}    %ради \endlasthead



% Русская традиция начертания греческих букв
\usepackage{upgreek} % прямые греческие ради русской традиции

%%% Микротипографика
%\ifnumequal{\value{draft}}{0}{% Только если у нас режим чистовика
%    \usepackage[final, babel, shrink=45]{microtype}[2016/05/14] % улучшает представление букв и слов в строках, может помочь при наличии отдельно висящих слов
%}{}

% Отметка о версии черновика на каждой странице
% Чтобы работало надо в своей локальной копии по инструкции
% https://www.ctan.org/pkg/gitinfo2 создать небходимые файлы в папке
% ./git/hooks
% If you’re familiar with tweaking git, you can probably work it out for
% yourself. If not, I suggest you follow these steps:
% 1. First, you need a git repository and working tree. For this example,
% let’s suppose that the root of the working tree is in ~/compsci
% 2. Copy the file post-xxx-sample.txt (which is in the same folder of
% your TEX distribution as this pdf) into the git hooks directory in your
% working copy. In our example case, you should end up with a file called
% ~/compsci/.git/hooks/post-checkout
% 3. If you’re using a unix-like system, don’t forget to make the file executable.
% Just how you do this is outside the scope of this manual, but one
% possible way is with commands such as this:
% chmod g+x post-checkout.
% 4. Test your setup with “git checkout master” (or another suitable branch
% name). This should generate copies of gitHeadInfo.gin in the directories
% you intended.
% 5. Now make two more copies of this file in the same directory (hooks),
% calling them post-commit and post-merge, and you’re done. As before,
% users of unix-like systems should ensure these files are marked as
% executable.
\ifnumequal{\value{draft}}{1}{% Черновик
	\IfFileExists{.git/gitHeadInfo.gin}{                                        
		\usepackage[mark,pcount]{gitinfo2}
		\renewcommand{\gitMark}{rev.\gitAbbrevHash\quad\gitCommitterEmail\quad\gitAuthorIsoDate}
		\renewcommand{\gitMarkFormat}{\rmfamily\color{Gray}\small\bfseries}
	}{}
}{}



\usepackage{minted}
\usepackage{tcolorbox}

\usepackage{pgfplotstable}
\usepackage{pgfplots}
\pgfplotsset{compat=1.12}

\tcbuselibrary{breakable,skins,minted}

\renewcommand{\listingscaption}{Листинг}

\newcommand{\putlisting}[1]{
	\tcbinputlisting{
		listing file=#1,
		minted language=vb.net,
		minted options={breaklines,fontsize=\footnotesize},% <-- put other minted options inside the brackets
		breakable,enhanced,% <-- put other tcolorbox options here
		listing only
	}
}


         % Пакеты общие для диссертации и автореферата
%%%%%%%%%%%%%%%%%%%%%%%%%%%%%%%%%%%%%%%%%%%%%%%%%%%%%%
%%%% Файл упрощённых настроек шаблона диссертации %%%%
%%%%%%%%%%%%%%%%%%%%%%%%%%%%%%%%%%%%%%%%%%%%%%%%%%%%%%

%%% Инициализирование переменных, не трогать!  %%%
\newcounter{intvl}
\newcounter{otstup}
\newcounter{contnumeq}
\newcounter{contnumfig}
\newcounter{contnumtab}
\newcounter{pgnum}
\newcounter{chapstyle}
\newcounter{headingdelim}
\newcounter{headingalign}
\newcounter{headingsize}
\newcounter{tabcap}
\newcounter{tablaba}
\newcounter{tabtita}
\newcounter{usefootcite}
%%%%%%%%%%%%%%%%%%%%%%%%%%%%%%%%%%%%%%%%%%%%%%%%%%

%%% Область упрощённого управления оформлением %%%

%% Интервал между заголовками и между заголовком и текстом
% Заголовки отделяют от текста сверху и снизу тремя интервалами (ГОСТ Р 7.0.11-2011, 5.3.5)
\setcounter{intvl}{3}               % Коэффициент кратности к размеру шрифта

%% Отступы у заголовков в тексте
\setcounter{otstup}{0}              % 0 --- без отступа; 1 --- абзацный отступ

%% Нумерация формул, таблиц и рисунков
\setcounter{contnumeq}{0}           % Нумерация формул: 0 --- пораздельно (во введении подряд, без номера раздела); 1 --- сквозная нумерация по всей диссертации
\setcounter{contnumfig}{0}          % Нумерация рисунков: 0 --- пораздельно (во введении подряд, без номера раздела); 1 --- сквозная нумерация по всей диссертации
\setcounter{contnumtab}{1}          % Нумерация таблиц: 0 --- пораздельно (во введении подряд, без номера раздела); 1 --- сквозная нумерация по всей диссертации

%% Оглавление
\setcounter{pgnum}{1}               % 0 --- номера страниц никак не обозначены; 1 --- Стр. над номерами страниц (дважды компилировать после изменения)
\settocdepth{subsection}            % до какого уровня подразделов выносить в оглавление
\setsecnumdepth{subsection}         % до какого уровня нумеровать подразделы


%% Текст и форматирование заголовков
\setcounter{chapstyle}{1}           % 0 --- разделы только под номером; 1 --- разделы с названием "Глава" перед номером
\setcounter{headingdelim}{1}        % 0 --- номер отделен пропуском в 1em или \quad; 1 --- номера разделов и приложений отделены точкой с пробелом, подразделы пропуском без точки; 2 --- номера разделов, подразделов и приложений отделены точкой с пробелом.

%% Выравнивание заголовков в тексте
\setcounter{headingalign}{0}        % 0 --- по центру; 1 --- по левому краю

%% Размеры заголовков в тексте
\setcounter{headingsize}{0}         % 0 --- по ГОСТ, все всегда 14 пт; 1 --- пропорционально изменяющийся размер в зависимости от базового шрифта

%% Подпись таблиц
\setcounter{tabcap}{0}              % 0 --- по ГОСТ, номер таблицы и название разделены тире, выровнены по левому краю, при необходимости на нескольких строках; 1 --- подпись таблицы не по ГОСТ, на двух и более строках, дальнейшие настройки: 
%Выравнивание первой строки, с подписью и номером
\setcounter{tablaba}{2}             % 0 --- по левому краю; 1 --- по центру; 2 --- по правому краю
%Выравнивание строк с самим названием таблицы
\setcounter{tabtita}{1}             % 0 --- по левому краю; 1 --- по центру; 2 --- по правому краю
%Разделитель записи «Таблица #» и названия таблицы
\newcommand{\tablabelsep}{ }

%% Подпись рисунков
%Разделитель записи «Рисунок #» и названия рисунка
\newcommand{\figlabelsep}{~\cyrdash\ } % (ГОСТ 2.105, 4.3.1) % "--- здесь не работает

%%% Цвета гиперссылок %%%
% Latex color definitions: http://latexcolor.com/
\definecolor{linkcolor}{rgb}{0.9,0,0}
\definecolor{citecolor}{rgb}{0,0.6,0}
\definecolor{urlcolor}{rgb}{0,0,1}
%\definecolor{linkcolor}{rgb}{0,0,0} %black
%\definecolor{citecolor}{rgb}{0,0,0} %black
%\definecolor{urlcolor}{rgb}{0,0,0} %black
   % Упрощённые настройки шаблона

%% Новые переменные, которые могут использоваться во всём проекте
% ГОСТ 7.0.11-2011
% 9.2 Оформление текста автореферата диссертации
% 9.2.1 Общая характеристика работы включает в себя следующие основные структурные
% элементы:
% актуальность темы исследования;
\newcommand{\actualityTXT}{Актуальность темы.}
% степень ее разработанности;
\newcommand{\progressTXT}{Степень разработанности темы.}
% цели и задачи;
\newcommand{\aimTXT}{Целью}
\newcommand{\tasksTXT}{задачи}
% научную новизну;
\newcommand{\noveltyTXT}{Научная новизна:}
% теоретическую и практическую значимость работы;
%\newcommand{\influenceTXT}{Теоретическая и практическая значимость}
% или чаще используют просто
\newcommand{\influenceTXT}{Практическая значимость}
% методологию и методы исследования;
\newcommand{\methodsTXT}{Mетодология и методы исследования.}
% положения, выносимые на защиту;
\newcommand{\defpositionsTXT}{Основные положения, выносимые на~защиту:}
% степень достоверности и апробацию результатов.
\newcommand{\reliabilityTXT}{Достоверность}
\newcommand{\probationTXT}{Апробация работы.}

\newcommand{\contributionTXT}{Личный вклад.}
\newcommand{\publicationsTXT}{Публикации.}


\newcommand{\authorbibtitle}{Публикации автора по теме диссертации}
\newcommand{\vakbibtitle}{В изданиях из списка ВАК РФ}
\newcommand{\notvakbibtitle}{В прочих изданиях}
\newcommand{\confbibtitle}{В сборниках трудов конференций}
\newcommand{\fullbibtitle}{Список литературы} % (ГОСТ Р 7.0.11-2011, 4)
         % Новые переменные, для всего проекта

%%%% Основные сведения %%%
\newcommand{\thesisAuthorLastName}{\todo{Хабибуллин}}
\newcommand{\thesisAuthorOtherNames}{\todo{Ринат Альфредович}}
\newcommand{\thesisAuthorInitials}{\todo{Р.\,А.}}
\newcommand{\Author}             % Диссертация, ФИО автора
{%
    \texorpdfstring{% \texorpdfstring takes two arguments and uses the first for (La)TeX and the second for pdf
        \thesisAuthorLastName~\thesisAuthorOtherNames% так будет отображаться на титульном листе или в тексте, где будет использоваться переменная
    }{%
        \thesisAuthorLastName, \thesisAuthorOtherNames% эта запись для свойств pdf-файла. В таком виде, если pdf будет обработан программами для сбора библиографических сведений, будет правильно представлена фамилия.
    }
}
\newcommand{\AuthorShort}        % Диссертация, ФИО автора инициалами
{\thesisAuthorInitials~\thesisAuthorLastName}
%\newcommand{\thesisUdk}                % Диссертация, УДК
%{\todo{xxx.xxx}}
\newcommand{\Title}              % Диссертация, название
{\todo{Исследования скважин и пластов}}
\newcommand{\thesisSpecialtyNumber}    % Диссертация, специальность, номер
{\todo{XX.XX.XX}}
\newcommand{\thesisSpecialtyTitle}     % Диссертация, специальность, название
{\todo{Название специальности}}
\newcommand{\thesisDegree}             % Диссертация, ученая степень
{\todo{кандидата физико-математических наук}}
\newcommand{\thesisDegreeShort}        % Диссертация, ученая степень, краткая запись
{\todo{канд. физ.-мат. наук}}
\newcommand{\thesisCity}               % Диссертация, город написания диссертации
{\todo{Город}}
\newcommand{\thesisYear}               % Диссертация, год написания диссертации
{\todo{20XX}}
\newcommand{\Organization}       % Диссертация, организация
{\todo{РГУ нефти и газа (НИУ) имени И.М.Губкина}}
\newcommand{\thesisOrganizationShort}  % Диссертация, краткое название организации для доклада
{\todo{НазУчДисРаб}}

\newcommand{\thesisInOrganization}     % Диссертация, организация в предложном падеже: Работа выполнена в ...
{\todo{учреждении, в~котором выполнялась данная диссертационная работа}}

\newcommand{\supervisorFio}            % Научный руководитель, ФИО
{\todo{Фамилия Имя Отчество}}
\newcommand{\supervisorRegalia}        % Научный руководитель, регалии
{\todo{уч. степень, уч. звание}}
\newcommand{\supervisorFioShort}       % Научный руководитель, ФИО
{\todo{И.\,О.~Фамилия}}
\newcommand{\supervisorRegaliaShort}   % Научный руководитель, регалии
{\todo{уч.~ст.,~уч.~зв.}}


\newcommand{\opponentOneFio}           % Оппонент 1, ФИО
{\todo{Фамилия Имя Отчество}}
\newcommand{\opponentOneRegalia}       % Оппонент 1, регалии
{\todo{доктор физико-математических наук, профессор}}
\newcommand{\opponentOneJobPlace}      % Оппонент 1, место работы
{\todo{Не очень длинное название для места работы}}
\newcommand{\opponentOneJobPost}       % Оппонент 1, должность
{\todo{старший научный сотрудник}}

\newcommand{\opponentTwoFio}           % Оппонент 2, ФИО
{\todo{Фамилия Имя Отчество}}
\newcommand{\opponentTwoRegalia}       % Оппонент 2, регалии
{\todo{кандидат физико-математических наук}}
\newcommand{\opponentTwoJobPlace}      % Оппонент 2, место работы
{\todo{Основное место работы c длинным длинным длинным длинным названием}}
\newcommand{\opponentTwoJobPost}       % Оппонент 2, должность
{\todo{старший научный сотрудник}}

\newcommand{\leadingOrganizationTitle} % Ведущая организация, дополнительные строки
{\todo{Федеральное государственное бюджетное образовательное учреждение высшего профессионального образования с~длинным длинным длинным длинным названием}}

\newcommand{\defenseDate}              % Защита, дата
{\todo{DD mmmmmmmm YYYY~г.~в~XX часов}}
\newcommand{\defenseCouncilNumber}     % Защита, номер диссертационного совета
{\todo{Д\,123.456.78}}
\newcommand{\defenseCouncilTitle}      % Защита, учреждение диссертационного совета
{\todo{Название учреждения}}
\newcommand{\defenseCouncilAddress}    % Защита, адрес учреждение диссертационного совета
{\todo{Адрес}}
\newcommand{\defenseCouncilPhone}      % Телефон для справок
{\todo{+7~(0000)~00-00-00}}

\newcommand{\defenseSecretaryFio}      % Секретарь диссертационного совета, ФИО
{\todo{Фамилия Имя Отчество}}
\newcommand{\defenseSecretaryRegalia}  % Секретарь диссертационного совета, регалии
{\todo{д-р~физ.-мат. наук}}            % Для сокращений есть ГОСТы, например: ГОСТ Р 7.0.12-2011 + http://base.garant.ru/179724/#block_30000

\newcommand{\synopsisLibrary}          % Автореферат, название библиотеки
{\todo{Название библиотеки}}
\newcommand{\synopsisDate}             % Автореферат, дата рассылки
{\todo{DD mmmmmmmm YYYY года}}

% To avoid conflict with beamer class use \providecommand
\providecommand{\keywords}%            % Ключевые слова для метаданных PDF диссертации и автореферата
{}
             % Основные сведения
%%% Кодировки и шрифты %%%

    \setmainlanguage[babelshorthands=true]{russian}    % Язык по-умолчанию русский с поддержкой приятных команд пакета babel
    \setotherlanguage{english}                         % Дополнительный язык = английский (в американской вариации по-умолчанию)

    % Проверка существования шрифтов. Недоступна в pdflatex
    \ifnumequal{\value{fontfamily}}{1}{
        \IfFontExistsTF{Times New Roman}{}{\setcounter{fontfamily}{0}}
    }{}
    \ifnumequal{\value{fontfamily}}{2}{
        \IfFontExistsTF{LiberationSerif}{}{\setcounter{fontfamily}{0}}
    }{}

    \ifnumequal{\value{fontfamily}}{0}{                    % Семейство шрифтов CMU. Используется как fallback
        \setmonofont{CMU Typewriter Text}                  % моноширинный шрифт
        \newfontfamily\cyrillicfonttt{CMU Typewriter Text} % моноширинный шрифт для кириллицы
        \defaultfontfeatures{Ligatures=TeX}                % стандартные лигатуры TeX, замены нескольких дефисов на тире и т. п. Настройки моноширинного шрифта должны идти до этой строки, чтобы при врезках кода программ в коде не применялись лигатуры и замены дефисов
        \setmainfont{CMU Serif}                            % Шрифт с засечками
        \newfontfamily\cyrillicfont{CMU Serif}             % Шрифт с засечками для кириллицы
        \setsansfont{CMU Sans Serif}                       % Шрифт без засечек
        \newfontfamily\cyrillicfontsf{CMU Sans Serif}      % Шрифт без засечек для кириллицы
    }

    \ifnumequal{\value{fontfamily}}{1}{                    % Семейство MS шрифтов
        \setmonofont{Courier New}                          % моноширинный шрифт
        \newfontfamily\cyrillicfonttt{Courier New}         % моноширинный шрифт для кириллицы
        \defaultfontfeatures{Ligatures=TeX}                % стандартные лигатуры TeX, замены нескольких дефисов на тире и т. п. Настройки моноширинного шрифта должны идти до этой строки, чтобы при врезках кода программ в коде не применялись лигатуры и замены дефисов
        \setmainfont{Times New Roman}                      % Шрифт с засечками
        \newfontfamily\cyrillicfont{Times New Roman}       % Шрифт с засечками для кириллицы
        \setsansfont{Arial}                                % Шрифт без засечек
        \newfontfamily\cyrillicfontsf{Arial}               % Шрифт без засечек для кириллицы
    }

    \ifnumequal{\value{fontfamily}}{2}{                    % Семейство шрифтов Liberation (https://pagure.io/liberation-fonts)
        \setmonofont{LiberationMono}[Scale=0.87] % моноширинный шрифт
        \newfontfamily\cyrillicfonttt{LiberationMono}[     % моноширинный шрифт для кириллицы
            Scale=0.87]
        \defaultfontfeatures{Ligatures=TeX}                % стандартные лигатуры TeX, замены нескольких дефисов на тире и т. п. Настройки моноширинного шрифта должны идти до этой строки, чтобы при врезках кода программ в коде не применялись лигатуры и замены дефисов
        \setmainfont{LiberationSerif}                      % Шрифт с засечками
        \newfontfamily\cyrillicfont{LiberationSerif}       % Шрифт с засечками для кириллицы
        \setsansfont{LiberationSans}                       % Шрифт без засечек
        \newfontfamily\cyrillicfontsf{LiberationSans}      % Шрифт без засечек для кириллицы
    }

% чтобы включить русские символы в формулах   http://www.linux.org.ru/forum/general/7659727
% может быть стоит и отключить, чтобы не злоупотреблять в процессе
% \DeclareSymbolFont{letters}{\encodingdefault}{\rmdefault}{m}{it}

\usepackage{unicode-math}                      %% использование Unicode-шрифтов для формул            % Определение шрифтов (частичное)
%%% Шаблон %%%
\DeclareRobustCommand{\todo}{\textcolor{red}}       % решаем проблему превращения названия цвета в результате \MakeUppercase, http://tex.stackexchange.com/a/187930, \DeclareRobustCommand protects \todo from expanding inside \MakeUppercase
\AtBeginDocument{%
    \setlength{\parindent}{2.5em}                   % Абзацный отступ. Должен быть одинаковым по всему тексту и равен пяти знакам (ГОСТ Р 7.0.11-2011, 5.3.7).
}

%%% Подписи %%%
\setlength{\abovecaptionskip}{0pt}   % Отбивка над подписью
\setlength{\belowcaptionskip}{0pt}   % Отбивка под подписью
\captionwidth{\linewidth}
\normalcaptionwidth

%%% Таблицы %%%
\ifnumequal{\value{tabcap}}{0}{%
    \newcommand{\tabcapalign}{\raggedright}  % по левому краю страницы или аналога parbox
    \renewcommand{\tablabelsep}{~\cyrdash\ } % тире как разделитель идентификатора с номером от наименования
    \newcommand{\tabtitalign}{}
}{%
    \ifnumequal{\value{tablaba}}{0}{%
        \newcommand{\tabcapalign}{\raggedright}  % по левому краю страницы или аналога parbox
    }{}

    \ifnumequal{\value{tablaba}}{1}{%
        \newcommand{\tabcapalign}{\centering}    % по центру страницы или аналога parbox
    }{}

    \ifnumequal{\value{tablaba}}{2}{%
        \newcommand{\tabcapalign}{\raggedleft}   % по правому краю страницы или аналога parbox
    }{}

    \ifnumequal{\value{tabtita}}{0}{%
        \newcommand{\tabtitalign}{\par\raggedright}  % по левому краю страницы или аналога parbox
    }{}

    \ifnumequal{\value{tabtita}}{1}{%
        \newcommand{\tabtitalign}{\par\centering}    % по центру страницы или аналога parbox
    }{}

    \ifnumequal{\value{tabtita}}{2}{%
        \newcommand{\tabtitalign}{\par\raggedleft}   % по правому краю страницы или аналога parbox
    }{}
}

\precaption{\tabcapalign} % всегда идет перед подписью или \legend
\captionnamefont{\normalfont\normalsize} % Шрифт надписи «Таблица #»; также определяет шрифт у \legend
\captiondelim{\tablabelsep} % разделитель идентификатора с номером от наименования
\captionstyle[\tabtitalign]{\tabtitalign}
\captiontitlefont{\normalfont\normalsize} % Шрифт с текстом подписи

%%% Рисунки %%%
\setfloatadjustment{figure}{%
    \setlength{\abovecaptionskip}{0pt}   % Отбивка над подписью
    \setlength{\belowcaptionskip}{0pt}   % Отбивка под подписью
    \precaption{} % всегда идет перед подписью или \legend
    \captionnamefont{\normalfont\normalsize} % Шрифт надписи «Рисунок #»; также определяет шрифт у \legend
    \captiondelim{\figlabelsep} % разделитель идентификатора с номером от наименования
    \captionstyle[\centering]{\centering} % Центрирование подписей, заданных командой \caption и \legend
    \captiontitlefont{\normalfont\normalsize} % Шрифт с текстом подписи
    \postcaption{} % всегда идет после подписи или \legend, и с новой строки
}

%%% Подписи подрисунков %%%
\newsubfloat{figure} % Включает возможность использовать подрисунки у окружений figure
\renewcommand{\thesubfigure}{\asbuk{subfigure}}           % Буквенные номера подрисунков
\subcaptionsize{\normalsize} % Шрифт подписи названий подрисунков (не отличается от основного)
\subcaptionlabelfont{\normalfont}
\subcaptionfont{\!\!) \normalfont} % Вот так тут добавили скобку после буквы.
\subcaptionstyle{\centering}
%\subcaptionsize{\fontsize{12pt}{13pt}\selectfont} % объявляем шрифт 12pt для использования в подписях, тут же надо интерлиньяж объявлять, если не наследуется

%%% Настройки гиперссылок %%%
\ifluatex
    \hypersetup{
        unicode,                % Unicode encoded PDF strings
    }
\fi

\hypersetup{
    linktocpage=true,           % ссылки с номера страницы в оглавлении, списке таблиц и списке рисунков
%    linktoc=all,                % both the section and page part are links
%    pdfpagelabels=false,        % set PDF page labels (true|false)
    plainpages=false,           % Forces page anchors to be named by the Arabic form  of the page number, rather than the formatted form
    colorlinks,                 % ссылки отображаются раскрашенным текстом, а не раскрашенным прямоугольником, вокруг текста
    linkcolor={linkcolor},      % цвет ссылок типа ref, eqref и подобных
    citecolor={citecolor},      % цвет ссылок-цитат
    urlcolor={urlcolor},        % цвет гиперссылок
%    hidelinks,                  % Hide links (removing color and border)
    pdftitle={\thesisTitle},    % Заголовок
%    pdfauthor={\thesisAuthor},  % Автор
%    pdfsubject={\thesisSpecialtyNumber\ \thesisSpecialtyTitle},      % Тема
%    pdfcreator={Создатель},     % Создатель, Приложение
%    pdfproducer={Производитель},% Производитель, Производитель PDF
%    pdfkeywords={\keywords},    % Ключевые слова
    pdflang={ru},
}
\ifnumequal{\value{draft}}{1}{% Черновик
    \hypersetup{
        draft,
    }
}{}

%%% Списки %%%
% Используем короткое тире (endash) для ненумерованных списков (ГОСТ 2.105-95, пункт 4.1.7, требует дефиса, но так лучше смотрится)
\renewcommand{\labelitemi}{\normalfont\bfseries{--}}

% Перечисление строчными буквами латинского алфавита (ГОСТ 2.105-95, 4.1.7)
%\renewcommand{\theenumi}{\alph{enumi}}
%\renewcommand{\labelenumi}{\theenumi)}

% Перечисление строчными буквами русского алфавита (ГОСТ 2.105-95, 4.1.7)
\makeatletter
\AddEnumerateCounter{\asbuk}{\russian@alph}{щ}      % Управляем списками/перечислениями через пакет enumitem, а он 'не знает' про asbuk, потому 'учим' его
\makeatother
%\renewcommand{\theenumi}{\asbuk{enumi}} %первый уровень нумерации
%\renewcommand{\labelenumi}{\theenumi)} %первый уровень нумерации
\renewcommand{\theenumii}{\asbuk{enumii}} %второй уровень нумерации
\renewcommand{\labelenumii}{\theenumii)} %второй уровень нумерации
\renewcommand{\theenumiii}{\arabic{enumiii}} %третий уровень нумерации
\renewcommand{\labelenumiii}{\theenumiii)} %третий уровень нумерации

\setlist{nosep,%                                    % Единый стиль для всех списков (пакет enumitem), без дополнительных интервалов.
    labelindent=\parindent,leftmargin=*%            % Каждый пункт, подпункт и перечисление записывают с абзацного отступа (ГОСТ 2.105-95, 4.1.8)
}

%%% Правильная нумерация приложений, рисунков и формул %%%
%% По ГОСТ 2.105, п. 4.3.8 Приложения обозначают заглавными буквами русского алфавита,
%% начиная с А, за исключением букв Ё, З, Й, О, Ч, Ь, Ы, Ъ.
%% Здесь также переделаны все нумерации русскими буквами.
\ifxetexorluatex
    \makeatletter
    \def\russian@Alph#1{\ifcase#1\or
       А\or Б\or В\or Г\or Д\or Е\or Ж\or
       И\or К\or Л\or М\or Н\or
       П\or Р\or С\or Т\or У\or Ф\or Х\or
       Ц\or Ш\or Щ\or Э\or Ю\or Я\else\xpg@ill@value{#1}{russian@Alph}\fi}
    \def\russian@alph#1{\ifcase#1\or
       а\or б\or в\or г\or д\or е\or ж\or
       и\or к\or л\or м\or н\or
       п\or р\or с\or т\or у\or ф\or х\or
       ц\or ш\or щ\or э\or ю\or я\else\xpg@ill@value{#1}{russian@alph}\fi}
    \def\cyr@Alph#1{\ifcase#1\or
        А\or Б\or В\or Г\or Д\or Е\or Ж\or
        И\or К\or Л\or М\or Н\or
        П\or Р\or С\or Т\or У\or Ф\or Х\or
        Ц\or Ш\or Щ\or Э\or Ю\or Я\else\xpg@ill@value{#1}{cyr@Alph}\fi}
    \def\cyr@alph#1{\ifcase#1\or
        а\or б\or в\or г\or д\or е\or ж\or
        и\or к\or л\or м\or н\or
        п\or р\or с\or т\or у\or ф\or х\or
        ц\or ш\or щ\or э\or ю\or я\else\xpg@ill@value{#1}{cyr@alph}\fi}
    \makeatother
\else
    \makeatletter
    \if@uni@ode
      \def\russian@Alph#1{\ifcase#1\or
        А\or Б\or В\or Г\or Д\or Е\or Ж\or
        И\or К\or Л\or М\or Н\or
        П\or Р\or С\or Т\or У\or Ф\or Х\or
        Ц\or Ш\or Щ\or Э\or Ю\or Я\else\@ctrerr\fi}
    \else
      \def\russian@Alph#1{\ifcase#1\or
        \CYRA\or\CYRB\or\CYRV\or\CYRG\or\CYRD\or\CYRE\or\CYRZH\or
        \CYRI\or\CYRK\or\CYRL\or\CYRM\or\CYRN\or
        \CYRP\or\CYRR\or\CYRS\or\CYRT\or\CYRU\or\CYRF\or\CYRH\or
        \CYRC\or\CYRSH\or\CYRSHCH\or\CYREREV\or\CYRYU\or
        \CYRYA\else\@ctrerr\fi}
    \fi
    \if@uni@ode
      \def\russian@alph#1{\ifcase#1\or
        а\or б\or в\or г\or д\or е\or ж\or
        и\or к\or л\or м\or н\or
        п\or р\or с\or т\or у\or ф\or х\or
        ц\or ш\or щ\or э\or ю\or я\else\@ctrerr\fi}
    \else
      \def\russian@alph#1{\ifcase#1\or
        \cyra\or\cyrb\or\cyrv\or\cyrg\or\cyrd\or\cyre\or\cyrzh\or
        \cyri\or\cyrk\or\cyrl\or\cyrm\or\cyrn\or
        \cyrp\or\cyrr\or\cyrs\or\cyrt\or\cyru\or\cyrf\or\cyrh\or
        \cyrc\or\cyrsh\or\cyrshch\or\cyrerev\or\cyryu\or
        \cyrya\else\@ctrerr\fi}
    \fi
    \makeatother
\fi
           % Стили общие для диссертации и автореферата
%%% Переопределение именований, если иначе не сработает %%%
%\gappto\captionsrussian{
%    \renewcommand{\chaptername}{Глава}
%    \renewcommand{\appendixname}{Приложение} % (ГОСТ Р 7.0.11-2011, 5.7)
%}

%%% Изображения %%%
%\graphicspath{{images/}{Dissertation/images/}}         % Пути к изображениям
\graphicspath{{pics/}}

%%% Интервалы %%%
%% По ГОСТ Р 7.0.11-2011, пункту 5.3.6 требуется полуторный интервал
%% Реализация средствами класса (на основе setspace) ближе к типографской классике.
%% И правит сразу и в таблицах (если со звёздочкой)
%\DoubleSpacing*     % Двойной интервал
\OnehalfSpacing*    % Полуторный интервал
%\setSpacing{1.42}   % Полуторный интервал, подобный Ворду (возможно, стоит включать вместе с предыдущей строкой)

%%% Макет страницы %%%
% Выставляем значения полей (ГОСТ 7.0.11-2011, 5.3.7)
\geometry{a4paper, top=2cm, bottom=2cm, left=2.5cm, right=1cm, nofoot, nomarginpar} %, heightrounded, showframe
\setlength{\topskip}{0pt}   %размер дополнительного верхнего поля
\setlength{\footskip}{12.3pt} % снимет warning, согласно https://tex.stackexchange.com/a/334346

%%% Выравнивание и переносы %%%
%% http://tex.stackexchange.com/questions/241343/what-is-the-meaning-of-fussy-sloppy-emergencystretch-tolerance-hbadness
%% http://www.latex-community.org/forum/viewtopic.php?p=70342#p70342
\tolerance 1414
\hbadness 1414
\emergencystretch 1.5em % В случае проблем регулировать в первую очередь
\hfuzz 0.3pt
\vfuzz \hfuzz
%\raggedbottom
%\sloppy                 % Избавляемся от переполнений
\clubpenalty=10000      % Запрещаем разрыв страницы после первой строки абзаца
\widowpenalty=10000     % Запрещаем разрыв страницы после последней строки абзаца
\brokenpenalty=4991     % Ограничение на разрыв страницы, если строка заканчивается переносом

%%% Блок управления параметрами для выравнивания заголовков в тексте %%%
\newlength{\otstuplen}
\setlength{\otstuplen}{\theotstup\parindent}
\ifnumequal{\value{headingalign}}{0}{% выравнивание заголовков в тексте
    \newcommand{\hdngalign}{\centering}                % по центру
    \newcommand{\hdngaligni}{}% по центру
    \setlength{\otstuplen}{0pt}
}{%
    \newcommand{\hdngalign}{}                 % по левому краю
    \newcommand{\hdngaligni}{\hspace{\otstuplen}}      % по левому краю
} % В обоих случаях вроде бы без переноса, как и надо (ГОСТ Р 7.0.11-2011, 5.3.5)

%%% Оглавление %%%
\renewcommand{\cftchapterdotsep}{\cftdotsep}                % отбивка точками до номера страницы начала главы/раздела

%% Переносить слова в заголовке не допускается (ГОСТ Р 7.0.11-2011, 5.3.5). Заголовки в оглавлении должны точно повторять заголовки в тексте (ГОСТ Р 7.0.11-2011, 5.2.3). Прямого указания на запрет переносов в оглавлении нет, но по той же логике невнесения искажений в смысл, лучше в оглавлении не переносить:
\setrmarg{2.55em plus1fil}                             %To have the (sectional) titles in the ToC, etc., typeset ragged right with no hyphenation
\renewcommand{\cftchapterpagefont}{\normalfont}        % нежирные номера страниц у глав в оглавлении
\renewcommand{\cftchapterleader}{\cftdotfill{\cftchapterdotsep}}% нежирные точки до номеров страниц у глав в оглавлении
%\renewcommand{\cftchapterfont}{}                       % нежирные названия глав в оглавлении

\ifnumgreater{\value{headingdelim}}{0}{%
    \renewcommand\cftchapteraftersnum{.\space}       % добавляет точку с пробелом после номера раздела в оглавлении
}{}
\ifnumgreater{\value{headingdelim}}{1}{%
    \renewcommand\cftsectionaftersnum{.\space}       % добавляет точку с пробелом после номера подраздела в оглавлении
    \renewcommand\cftsubsectionaftersnum{.\space}    % добавляет точку с пробелом после номера подподраздела в оглавлении
    \renewcommand\cftsubsubsectionaftersnum{.\space} % добавляет точку с пробелом после номера подподподраздела в оглавлении
    \AtBeginDocument{% без этого polyglossia сама всё переопределяет
        \setsecnumformat{\csname the#1\endcsname.\space}
    }
}{%
    \AtBeginDocument{% без этого polyglossia сама всё переопределяет
        \setsecnumformat{\csname the#1\endcsname\quad}
    }
}

\renewcommand*{\cftappendixname}{\appendixname\space} % Слово Приложение в оглавлении

%%% Колонтитулы %%%
% Порядковый номер страницы печатают на середине верхнего поля страницы (ГОСТ Р 7.0.11-2011, 5.3.8)
\makeevenhead{plain}{}{\thepage}{}
\makeoddhead{plain}{}{\thepage}{}
\makeevenfoot{plain}{}{}{}
\makeoddfoot{plain}{}{}{}
\pagestyle{plain}

%%% добавить Стр. над номерами страниц в оглавлении
%%% http://tex.stackexchange.com/a/306950
\newif\ifendTOC

\newcommand*{\tocheader}{
\ifnumequal{\value{pgnum}}{1}{%
    \ifendTOC\else\hbox to \linewidth%
      {\noindent{}~\hfill{Стр.}}\par%
      \ifnumless{\value{page}}{3}{}{%
        \vspace{0.5\onelineskip}
      }
      \afterpage{\tocheader}
    \fi%
}{}%
}%

%%% Оформление заголовков глав, разделов, подразделов %%%
%% Работа должна быть выполнена ... размером шрифта 12-14 пунктов (ГОСТ Р 7.0.11-2011, 5.3.8). То есть не должно быть надписей шрифтом более 14. Так и поставим.
%% Эти установки будут давать одинаковый результат независимо от выбора базовым шрифтом 12 пт или 14 пт
\newcommand{\basegostsectionfont}{\fontsize{14pt}{16pt}\selectfont\bfseries}

\makechapterstyle{thesisgost}{%
    \chapterstyle{default}
    \setlength{\beforechapskip}{0pt}
    \setlength{\midchapskip}{0pt}
    \setlength{\afterchapskip}{\theintvl\curtextsize}
    \renewcommand*{\chapnamefont}{\basegostsectionfont}
    \renewcommand*{\chapnumfont}{\basegostsectionfont}
    \renewcommand*{\chaptitlefont}{\basegostsectionfont}
    \renewcommand*{\chapterheadstart}{}
    \ifnumgreater{\value{headingdelim}}{0}{%
        \renewcommand*{\afterchapternum}{.\space}   % добавляет точку с пробелом после номера раздела
    }{%
        \renewcommand*{\afterchapternum}{\quad}     % добавляет \quad после номера раздела
    }
    \renewcommand*{\printchapternum}{\hdngaligni\hdngalign\chapnumfont \thechapter}
    \renewcommand*{\printchaptername}{}
    \renewcommand*{\printchapternonum}{\hdngaligni\hdngalign}
}

\makeatletter
\makechapterstyle{thesisgostchapname}{%
    \chapterstyle{thesisgost}
    \renewcommand*{\printchapternum}{\chapnumfont \thechapter}
    \renewcommand*{\printchaptername}{\hdngaligni\hdngalign\chapnamefont \@chapapp} %
}
\makeatother

\chapterstyle{thesisgost}

\setsecheadstyle{\basegostsectionfont\hdngalign}
\setsecindent{\otstuplen}

\setsubsecheadstyle{\basegostsectionfont\hdngalign}
\setsubsecindent{\otstuplen}

\setsubsubsecheadstyle{\basegostsectionfont\hdngalign}
\setsubsubsecindent{\otstuplen}

\sethangfrom{\noindent #1} %все заголовки подразделов центрируются с учетом номера, как block

\ifnumequal{\value{chapstyle}}{1}{%
    \chapterstyle{thesisgostchapname}
    \renewcommand*{\cftchaptername}{\chaptername\space} % будет вписано слово Глава перед каждым номером раздела в оглавлении
}{}%

%%% Интервалы между заголовками
\setbeforesecskip{\theintvl\curtextsize}% Заголовки отделяют от текста сверху и снизу тремя интервалами (ГОСТ Р 7.0.11-2011, 5.3.5).
\setaftersecskip{\theintvl\curtextsize}
\setbeforesubsecskip{\theintvl\curtextsize}
\setaftersubsecskip{\theintvl\curtextsize}
\setbeforesubsubsecskip{\theintvl\curtextsize}
\setaftersubsubsecskip{\theintvl\curtextsize}

%%% Блок дополнительного управления размерами заголовков
\ifnumequal{\value{headingsize}}{1}{% Пропорциональные заголовки и базовый шрифт 14 пт
    \renewcommand{\basegostsectionfont}{\large\bfseries}
    \renewcommand*{\chapnamefont}{\Large\bfseries}
    \renewcommand*{\chapnumfont}{\Large\bfseries}
    \renewcommand*{\chaptitlefont}{\Large\bfseries}
}{}

%%% Счётчики %%%

%% Упрощённые настройки шаблона диссертации: нумерация формул, таблиц, рисунков
\ifnumequal{\value{contnumeq}}{1}{%
    \counterwithout{equation}{chapter} % Убираем связанность номера формулы с номером главы/раздела
}{}
\ifnumequal{\value{contnumfig}}{1}{%
    \counterwithout{figure}{chapter}   % Убираем связанность номера рисунка с номером главы/раздела
}{}
\ifnumequal{\value{contnumtab}}{1}{%
    \counterwithout{table}{chapter}    % Убираем связанность номера таблицы с номером главы/раздела
}{}


%%http://www.linux.org.ru/forum/general/6993203#comment-6994589 (используется totcount)
\makeatletter
\def\formbytotal#1#2#3#4#5{%
    \newcount\@c
    \@c\totvalue{#1}\relax
    \newcount\@last
    \newcount\@pnul
    \@last\@c\relax
    \divide\@last 10
    \@pnul\@last\relax
    \divide\@pnul 10
    \multiply\@pnul-10
    \advance\@pnul\@last
    \multiply\@last-10
    \advance\@last\@c
    \total{#1}~#2%
    \ifnum\@pnul=1#5\else%
    \ifcase\@last#5\or#3\or#4\or#4\or#4\else#5\fi
    \fi
}
\makeatother

\AtBeginDocument{
%% регистрируем счётчики в системе totcounter
    \regtotcounter{totalcount@figure}
    \regtotcounter{totalcount@table}       % Если иным способом поставить в преамбуле то ошибка в числе таблиц
    \regtotcounter{TotPages}               % Если иным способом поставить в преамбуле то ошибка в числе страниц
}

%%% Правильная нумерация приложений %%%
%% По ГОСТ 2.105, п. 4.3.8 Приложения обозначают заглавными буквами русского алфавита,
%% начиная с А, за исключением букв Ё, З, Й, О, Ч, Ь, Ы, Ъ.
%% Здесь также переделаны все нумерации русскими буквами.

    \makeatletter
    \def\russian@Alph#1{\ifcase#1\or
       А\or Б\or В\or Г\or Д\or Е\or Ж\or
       И\or К\or Л\or М\or Н\or
       П\or Р\or С\or Т\or У\or Ф\or Х\or
       Ц\or Ш\or Щ\or Э\or Ю\or Я\else\xpg@ill@value{#1}{russian@Alph}\fi}
    \def\russian@alph#1{\ifcase#1\or
       а\or б\or в\or г\or д\or е\or ж\or
       и\or к\or л\or м\or н\or
       п\or р\or с\or т\or у\or ф\or х\or
       ц\or ш\or щ\or э\or ю\or я\else\xpg@ill@value{#1}{russian@alph}\fi}
    \makeatother
  		 % Стили для диссертации
% для вертикального центрирования ячеек в tabulary
\def\zz{\ifx\[$\else\aftergroup\zzz\fi}
%$ \] % <-- чиним подсветку синтаксиса в некоторых редакторах
\def\zzz{\setbox0\lastbox
\dimen0\dimexpr\extrarowheight + \ht0-\dp0\relax
\setbox0\hbox{\raise-.5\dimen0\box0}%
\ht0=\dimexpr\ht0+\extrarowheight\relax
\dp0=\dimexpr\dp0+\extrarowheight\relax 
\box0
}




%Общие счётчики окружений листингов
%http://tex.stackexchange.com/questions/145546/how-to-make-figure-and-listing-share-their-counter
% Если смешивать плавающие и не плавающие окружения, то могут быть проблемы с нумерацией
\makeatletter
\AtBeginDocument{%
    \let\c@ListingEnv\c@lstlisting
    \let\theListingEnv\thelstlisting
    \let\ftype@lstlisting\ftype@ListingEnv % give the floats the same precedence
}
\makeatother

% значок С++ — используйте команду \cpp
\newcommand{\cpp}{%
    C\nolinebreak\hspace{-.05em}%
    \raisebox{.2ex}{+}\nolinebreak\hspace{-.10em}%
    \raisebox{.2ex}{+}%
}

%%%  Чересстрочное форматирование таблиц
%% http://tex.stackexchange.com/questions/278362/apply-italic-formatting-to-every-other-row
\newcounter{rowcnt}
\newcommand\altshape{\ifnumodd{\value{rowcnt}}{\color{red}}{\vspace*{-1ex}\itshape}}
% \AtBeginEnvironment{tabular}{\setcounter{rowcnt}{1}}
% \AtEndEnvironment{tabular}{\setcounter{rowcnt}{0}}

%%% Ради примера во второй главе
\let\originalepsilon\epsilon
\let\originalphi\phi
\let\originalkappa\kappa
\let\originalle\le
\let\originalleq\leq
\let\originalge\ge
\let\originalgeq\geq
\let\originalemptyset\emptyset
\let\originaltan\tan
\let\originalcot\cot
\let\originalcsc\csc

%%% Русская традиция начертания математических знаков
\renewcommand{\le}{\ensuremath{\leqslant}}
\renewcommand{\leq}{\ensuremath{\leqslant}}
\renewcommand{\ge}{\ensuremath{\geqslant}}
\renewcommand{\geq}{\ensuremath{\geqslant}}
\renewcommand{\emptyset}{\varnothing}

%%% Русская традиция начертания математических функций (на случай копирования из зарубежных источников)
\renewcommand{\tan}{\operatorname{tg}}
\renewcommand{\cot}{\operatorname{ctg}}
\renewcommand{\csc}{\operatorname{cosec}}

%%% Русская традиция начертания греческих букв (греческие буквы вертикальные, через пакет upgreek)
\renewcommand{\epsilon}{\ensuremath{\upvarepsilon}}   %  русская традиция записи
\renewcommand{\phi}{\ensuremath{\upvarphi}}
%\renewcommand{\kappa}{\ensuremath{\varkappa}}
\renewcommand{\alpha}{\upalpha}
\renewcommand{\beta}{\upbeta}
\renewcommand{\gamma}{\upgamma}
\renewcommand{\delta}{\updelta}
\renewcommand{\varepsilon}{\upvarepsilon}
\renewcommand{\zeta}{\upzeta}
\renewcommand{\eta}{\upeta}
\renewcommand{\theta}{\uptheta}
\renewcommand{\vartheta}{\upvartheta}
\renewcommand{\iota}{\upiota}
\renewcommand{\kappa}{\upkappa}
\renewcommand{\lambda}{\uplambda}
\renewcommand{\mu}{\upmu}
\renewcommand{\nu}{\upnu}
\renewcommand{\xi}{\upxi}
\renewcommand{\pi}{\uppi}
\renewcommand{\varpi}{\upvarpi}
\renewcommand{\rho}{\uprho}
%\renewcommand{\varrho}{\upvarrho}
\renewcommand{\sigma}{\upsigma}
%\renewcommand{\varsigma}{\upvarsigma}
\renewcommand{\tau}{\uptau}
\renewcommand{\upsilon}{\upupsilon}
\renewcommand{\varphi}{\upvarphi}
\renewcommand{\chi}{\upchi}
\renewcommand{\psi}{\uppsi}
\renewcommand{\omega}{\upomega}
 		 % Стили для специфических пользовательских задач

%%% Библиография %%%
%%% Реализация библиографии пакетами biblatex и biblatex-gost с использованием движка biber %%%

\usepackage{csquotes} % biblatex рекомендует его подключать. Пакет для оформления сложных блоков цитирования.
%%% Загрузка пакета с основными настройками %%%
\makeatletter
\ifnumequal{\value{draft}}{0}{% Чистовик
	\usepackage[%
	backend=biber,% движок
	bibencoding=utf8,% кодировка bib файла
	sorting=none,% настройка сортировки списка литературы
	style=gost-numeric,% стиль цитирования и библиографии (по ГОСТ)
	language=autobib,% получение языка из babel/polyglossia, default: autobib % если ставить autocite или auto, то цитаты в тексте с указанием страницы, получат указание страницы на языке оригинала
	autolang=other,% многоязычная библиография
	clearlang=true,% внутренний сброс поля language, если он совпадает с языком из babel/polyglossia
	defernumbers=true,% нумерация проставляется после двух компиляций, зато позволяет выцеплять библиографию по ключевым словам и нумеровать не из большего списка
	sortcites=true,% сортировать номера затекстовых ссылок при цитировании (если в квадратных скобках несколько ссылок, то отображаться будут отсортированно, а не абы как)
	doi=false,% Показывать или нет ссылки на DOI
	isbn=false,% Показывать или нет ISBN, ISSN, ISRN
	]{biblatex}[2016/09/17]
	\ltx@iffilelater{biblatex-gost.def}{2017/05/03}%
	{\toggletrue{bbx:gostbibliography}%
		\renewcommand*{\revsdnamepunct}{\addcomma}}{}
}{%Черновик
	\usepackage[%
	backend=biber,% движок
	bibencoding=utf8,% кодировка bib файла
	sorting=none,% настройка сортировки списка литературы
	% defernumbers=true, % откомментируйте, если требуется правильная нумерация ссылок на литературу в режиме черновика. Замедляет сборку
	]{biblatex}[2016/09/17]%
}
\makeatother

\ifxetexorluatex
\else
% Исправление случая неподдержки знака номера в pdflatex
\DefineBibliographyStrings{russian}{number={\textnumero}}
\fi

\ifsynopsis
\ifnumgreater{\value{usefootcite}}{0}{
	\ExecuteBibliographyOptions{autocite=footnote}
	\newbibmacro*{cite:full}{%
		\printtext[bibhypertarget]{%
			\usedriver{%
				\DeclareNameAlias{sortname}{default}%
			}{%
				\thefield{entrytype}%
			}%
		}%
		\usebibmacro{shorthandintro}%
	}
	\DeclareCiteCommand{\smartcite}[\mkbibfootnote]{%
		\usebibmacro{prenote}%
	}{%
		\usebibmacro{citeindex}%
		\usebibmacro{cite:full}%
	}{%
		\multicitedelim%
	}{%
		\usebibmacro{postnote}%
	}
}{}
\fi


%%% Подключение файлов bib %%%
\addbibresource[label=other]{../biblio/othercites.bib}
%\addbibresource[label=vak]{biblio/authorpapersVAK.bib}
\addbibresource[label=papers]{../biblio/biblio_papers.bib}
\addbibresource[label=books]{../biblio/biblio_books.bib}
\addbibresource[label=links]{../biblio/biblio_links.bib}
\addbibresource[label=dis]{../biblio/biblio_dis.bib}

%http://tex.stackexchange.com/a/141831/79756
%There is a way to automatically map the language field to the langid field. The following lines in the preamble should be enough to do that.
%This command will copy the language field into the langid field and will then delete the contents of the language field. The language field will only be deleted if it was successfully copied into the langid field.
\DeclareSourcemap{ %модификация bib файла перед тем, как им займётся biblatex
	\maps{
		\map{% перекидываем значения полей language в поля langid, которыми пользуется biblatex
			\step[fieldsource=language, fieldset=langid, origfieldval, final]
			\step[fieldset=language, null]
		}
		\map{% перекидываем значения полей numpages в поля pagetotal, которыми пользуется biblatex
			\step[fieldsource=numpages, fieldset=pagetotal, origfieldval, final]
			\step[fieldset=numpages, null]
		}
		\map{% перекидываем значения полей pagestotal в поля pagetotal, которыми пользуется biblatex
			\step[fieldsource=pagestotal, fieldset=pagetotal, origfieldval, final]
			\step[fieldset=pagestotal, null]
		}
		\map[overwrite]{% перекидываем значения полей shortjournal, если они есть, в поля journal, которыми пользуется biblatex
			\step[fieldsource=shortjournal, final]
			\step[fieldset=journal, origfieldval]
			\step[fieldset=shortjournal, null]
		}
		\map[overwrite]{% перекидываем значения полей shortbooktitle, если они есть, в поля booktitle, которыми пользуется biblatex
			\step[fieldsource=shortbooktitle, final]
			\step[fieldset=booktitle, origfieldval]
			\step[fieldset=shortbooktitle, null]
		}
		\map{% если в поле medium написано "Электронный ресурс", то устанавливаем поле media, которым пользуется biblatex, в значение eresource.
			\step[fieldsource=medium,
			match=\regexp{Электронный\s+ресурс},
			final]
			\step[fieldset=media, fieldvalue=eresource]
			\step[fieldset=medium, null]
		}
		\map{% использование media=text по умолчанию
			\step[fieldset=media, fieldvalue=text]
		}
		\map[overwrite]{% стираем значения всех полей issn
			\step[fieldset=issn, null]
		}
		\map[overwrite]{% стираем значения всех полей abstract, поскольку ими не пользуемся, а там бывают "неприятные" латеху символы
			\step[fieldsource=abstract]
			\step[fieldset=abstract,null]
		}
		\map[overwrite]{ % переделка формата записи даты
			\step[fieldsource=urldate,
			match=\regexp{([0-9]{2})\.([0-9]{2})\.([0-9]{4})},
			replace={$3-$2-$1$4}, % $4 вставлен исключительно ради нормальной работы программ подсветки синтаксиса, которые некорректно обрабатывают $ в таких конструкциях
			final]
		}
		\map[overwrite]{ % стираем ключевые слова
			\step[fieldsource=keywords]
			\step[fieldset=keywords,null]
		}
		% реализация foreach различается для biblatex v3.12 и v3.13.
		% Для версии v3.13 эта конструкция заменяет последующие 5 структур map
		% \map[overwrite,foreach={authorvak,authorscopus,authorwos,authorconf,authorother}]{ % записываем информацию о типе публикации в ключевые слова
		%     \step[fieldsource=$MAPLOOP,final=true]
		%     \step[fieldset=keywords,fieldvalue={,biblio$MAPLOOP},append=true]
		% }
		\map[overwrite]{ % записываем информацию о типе публикации в ключевые слова
			\step[fieldsource=authorvak,final=true]
			\step[fieldset=keywords,fieldvalue={,biblioauthorvak},append=true]
		}
		\map[overwrite]{ % записываем информацию о типе публикации в ключевые слова
			\step[fieldsource=authorscopus,final=true]
			\step[fieldset=keywords,fieldvalue={,biblioauthorscopus},append=true]
		}
		\map[overwrite]{ % записываем информацию о типе публикации в ключевые слова
			\step[fieldsource=authorwos,final=true]
			\step[fieldset=keywords,fieldvalue={,biblioauthorwos},append=true]
		}
		\map[overwrite]{ % записываем информацию о типе публикации в ключевые слова
			\step[fieldsource=authorconf,final=true]
			\step[fieldset=keywords,fieldvalue={,biblioauthorconf},append=true]
		}
		\map[overwrite]{ % записываем информацию о типе публикации в ключевые слова
			\step[fieldsource=authorother,final=true]
			\step[fieldset=keywords,fieldvalue={,biblioauthorother},append=true]
		}
		\map[overwrite]{ % добавляем ключевые слова, чтобы различать источники
			\perdatasource{biblio/external.bib}
			\step[fieldset=keywords, fieldvalue={,biblioexternal},append=true]
		}
		\map[overwrite]{ % добавляем ключевые слова, чтобы различать источники
			\perdatasource{biblio/author.bib}
			\step[fieldset=keywords, fieldvalue={,biblioauthor},append=true]
		}
		\map[overwrite]{ % добавляем ключевые слова, чтобы различать источники
			\step[fieldset=keywords, fieldvalue={,bibliofull},append=true]
		}
		%        \map[overwrite]{% стираем значения всех полей series
		%            \step[fieldset=series, null]
		%        }
		\map[overwrite]{% перекидываем значения полей howpublished в поля organization для типа online
			\step[typesource=online, typetarget=online, final]
			\step[fieldsource=howpublished, fieldset=organization, origfieldval]
			\step[fieldset=howpublished, null]
		}
		% Так отключаем [Электронный ресурс]
		%        \map[overwrite]{% стираем значения всех полей media=eresource
		%            \step[fieldsource=media,
		%            match={eresource},
		%            final]
		%            \step[fieldset=media, null]
		%        }
	}
}

\ifsynopsis
\else
\DeclareSourcemap{ %модификация bib файла перед тем, как им займётся biblatex
	\maps{
		\map[overwrite]{% стираем значения всех полей addendum
			\perdatasource{biblio/author.bib}
			\step[fieldset=addendum, null] %чтобы избавиться от информации об объёме авторских статей, в отличие от автореферата
		}
	}
}
\fi

\defbibfilter{vakscopuswos}{%
	keyword=biblioauthorvak or keyword=biblioauthorscopus or keyword=biblioauthorwos
}

\defbibfilter{scopuswos}{%
	keyword=biblioauthorscopus or keyword=biblioauthorwos
}

%%% Убираем неразрывные пробелы перед двоеточием и точкой с запятой %%%
%\makeatletter
%\ifnumequal{\value{draft}}{0}{% Чистовик
%    \renewcommand*{\addcolondelim}{%
%      \begingroup%
%      \def\abx@colon{%
%        \ifdim\lastkern>\z@\unkern\fi%
%        \abx@puncthook{:}\space}%
%      \addcolon%
%      \endgroup}
%
%    \renewcommand*{\addsemicolondelim}{%
%      \begingroup%
%      \def\abx@semicolon{%
%        \ifdim\lastkern>\z@\unkern\fi%
%        \abx@puncthook{;}\space}%
%      \addsemicolon%
%      \endgroup}
%}{}
%\makeatother

%%% Правка записей типа thesis, чтобы дважды не писался автор
%\ifnumequal{\value{draft}}{0}{% Чистовик
%\DeclareBibliographyDriver{thesis}{%
%  \usebibmacro{bibindex}%
%  \usebibmacro{begentry}%
%  \usebibmacro{heading}%
%  \newunit
%  \usebibmacro{author}%
%  \setunit*{\labelnamepunct}%
%  \usebibmacro{thesistitle}%
%  \setunit{\respdelim}%
%  %\printnames[last-first:full]{author}%Вот эту строчку нужно убрать, чтобы автор диссертации не дублировался
%  \newunit\newblock
%  \printlist[semicolondelim]{specdata}%
%  \newunit
%  \usebibmacro{institution+location+date}%
%  \newunit\newblock
%  \usebibmacro{chapter+pages}%
%  \newunit
%  \printfield{pagetotal}%
%  \newunit\newblock
%  \usebibmacro{doi+eprint+url+note}%
%  \newunit\newblock
%  \usebibmacro{addendum+pubstate}%
%  \setunit{\bibpagerefpunct}\newblock
%  \usebibmacro{pageref}%
%  \newunit\newblock
%  \usebibmacro{related:init}%
%  \usebibmacro{related}%
%  \usebibmacro{finentry}}
%}{}

%\newbibmacro{string+doi}[1]{% новая макрокоманда на простановку ссылки на doi
%    \iffieldundef{doi}{#1}{\href{http://dx.doi.org/\thefield{doi}}{#1}}}

%\ifnumequal{\value{draft}}{0}{% Чистовик
%\renewcommand*{\mkgostheading}[1]{\usebibmacro{string+doi}{#1}} % ссылка на doi с авторов. стоящих впереди записи
%\renewcommand*{\mkgostheading}[1]{#1} % только лишь убираем курсив с авторов
%}{}
%\DeclareFieldFormat{title}{\usebibmacro{string+doi}{#1}} % ссылка на doi с названия работы
%\DeclareFieldFormat{journaltitle}{\usebibmacro{string+doi}{#1}} % ссылка на doi с названия журнала
%%% Тире как разделитель в библиографии традиционной руской длины:
\renewcommand*{\newblockpunct}{\addperiod\addnbspace\cyrdash\space\bibsentence}
%%% Убрать тире из разделителей элементов в библиографии:
%\renewcommand*{\newblockpunct}{%
%    \addperiod\space\bibsentence}%block punct.,\bibsentence is for vol,etc.

%%% Возвращаем запись «Режим доступа» %%%
%\DefineBibliographyStrings{english}{%
%    urlfrom = {Mode of access}
%}
%\DeclareFieldFormat{url}{\bibstring{urlfrom}\addcolon\space\url{#1}}

%%% В списке литературы обозначение одной буквой диапазона страниц англоязычного источника %%%
\DefineBibliographyStrings{english}{%
	pages = {p\adddot} %заглавность буквы затем по месту определяется работой самого biblatex
}

%%% В ссылке на источник в основном тексте с указанием конкретной страницы обозначение одной большой буквой %%%
%\DefineBibliographyStrings{russian}{%
%    page = {C\adddot}
%}

%%% Исправление длины тире в диапазонах %%%
% \cyrdash --- тире «русской» длины, \textendash --- en-dash
\DefineBibliographyExtras{russian}{%
	\protected\def\bibrangedash{%
		\cyrdash\penalty\value{abbrvpenalty}}% almost unbreakable dash
	\protected\def\bibdaterangesep{\bibrangedash}%тире для дат
}
\DefineBibliographyExtras{english}{%
	\protected\def\bibrangedash{%
		\cyrdash\penalty\value{abbrvpenalty}}% almost unbreakable dash
	\protected\def\bibdaterangesep{\bibrangedash}%тире для дат
}

%Set higher penalty for breaking in number, dates and pages ranges
\setcounter{abbrvpenalty}{10000} % default is \hyphenpenalty which is 12

%Set higher penalty for breaking in names
\setcounter{highnamepenalty}{10000} % If you prefer the traditional BibTeX behavior (no linebreaks at highnamepenalty breakpoints), set it to ‘infinite’ (10 000 or higher).
\setcounter{lownamepenalty}{10000}

%%% Set low penalties for breaks at uppercase letters and lowercase letters
%\setcounter{biburllcpenalty}{500} %управляет разрывами ссылок после маленьких букв RTFM biburllcpenalty
%\setcounter{biburlucpenalty}{3000} %управляет разрывами ссылок после больших букв, RTFM biburlucpenalty

%%% Список литературы с красной строки (без висячего отступа) %%%
%\defbibenvironment{bibliography} % переопределяем окружение библиографии из gost-numeric.bbx пакета biblatex-gost
%  {\list
%     {\printtext[labelnumberwidth]{%
%       \printfield{prefixnumber}%
%       \printfield{labelnumber}}}
%     {%
%      \setlength{\labelwidth}{\labelnumberwidth}%
%      \setlength{\leftmargin}{0pt}% default is \labelwidth
%      \setlength{\labelsep}{\widthof{\ }}% Управляет длиной отступа после точки % default is \biblabelsep
%      \setlength{\itemsep}{\bibitemsep}% Управление дополнительным вертикальным разрывом между записями. \bibitemsep по умолчанию соответствует \itemsep списков в документе.
%      \setlength{\itemindent}{\bibhang}% Пользуемся тем, что \bibhang по умолчанию принимает значение \parindent (абзацного отступа), который переназначен в styles.tex
%      \addtolength{\itemindent}{\labelwidth}% Сдвигаем правее на величину номера с точкой
%      \addtolength{\itemindent}{\labelsep}% Сдвигаем ещё правее на отступ после точки
%      \setlength{\parsep}{\bibparsep}%
%     }%
%      \renewcommand*{\makelabel}[1]{\hss##1}%
%  }
%  {\endlist}
%  {\item}

%%% Макросы автоматического подсчёта количества авторских публикаций.
% Печатают невидимую (пустую) библиографию, считая количество источников.
% http://tex.stackexchange.com/a/66851/79756
%
\makeatletter
\newtotcounter{citenum}
\defbibenvironment{counter}
{\setcounter{citenum}{0}\renewcommand{\blx@driver}[1]{}} % begin code: убирает весь выводимый текст
{} % end code
{\stepcounter{citenum}} % item code: cчитает "печатаемые в библиографию" источники

\newtotcounter{citeauthorvak}
\defbibenvironment{countauthorvak}
{\setcounter{citeauthorvak}{0}\renewcommand{\blx@driver}[1]{}}
{}
{\stepcounter{citeauthorvak}}

\newtotcounter{citeauthorscopus}
\defbibenvironment{countauthorscopus}
{\setcounter{citeauthorscopus}{0}\renewcommand{\blx@driver}[1]{}}
{}
{\stepcounter{citeauthorscopus}}

\newtotcounter{citeauthorwos}
\defbibenvironment{countauthorwos}
{\setcounter{citeauthorwos}{0}\renewcommand{\blx@driver}[1]{}}
{}
{\stepcounter{citeauthorwos}}

\newtotcounter{citeauthorother}
\defbibenvironment{countauthorother}
{\setcounter{citeauthorother}{0}\renewcommand{\blx@driver}[1]{}}
{}
{\stepcounter{citeauthorother}}

\newtotcounter{citeauthorconf}
\defbibenvironment{countauthorconf}
{\setcounter{citeauthorconf}{0}\renewcommand{\blx@driver}[1]{}}
{}
{\stepcounter{citeauthorconf}}

\newtotcounter{citeauthor}
\defbibenvironment{countauthor}
{\setcounter{citeauthor}{0}\renewcommand{\blx@driver}[1]{}}
{}
{\stepcounter{citeauthor}}

\newtotcounter{citeauthorvakscopuswos}
\defbibenvironment{countauthorvakscopuswos}
{\setcounter{citeauthorvakscopuswos}{0}\renewcommand{\blx@driver}[1]{}}
{}
{\stepcounter{citeauthorvakscopuswos}}

\newtotcounter{citeauthorscopuswos}
\defbibenvironment{countauthorscopuswos}
{\setcounter{citeauthorscopuswos}{0}\renewcommand{\blx@driver}[1]{}}
{}
{\stepcounter{citeauthorscopuswos}}

\newtotcounter{citeexternal}
\defbibenvironment{countexternal}
{\setcounter{citeexternal}{0}\renewcommand{\blx@driver}[1]{}}
{}
{\stepcounter{citeexternal}}
\makeatother

\defbibheading{nobibheading}{} % пустой заголовок, для подсчёта публикаций с помощью невидимой библиографии
\defbibheading{pubgroup}{\section*{#1}} % обычный стиль, заголовок-секция
\defbibheading{pubsubgroup}{\noindent\textbf{#1}} % для подразделов "по типу источника"

%%%Сортировка списка литературы Русский-Английский (предварительно удалить dissertation.bbl) (начало)
%%%Источник: https://github.com/odomanov/biblatex-gost/wiki/%D0%9A%D0%B0%D0%BA-%D1%81%D0%B4%D0%B5%D0%BB%D0%B0%D1%82%D1%8C,-%D1%87%D1%82%D0%BE%D0%B1%D1%8B-%D1%80%D1%83%D1%81%D1%81%D0%BA%D0%BE%D1%8F%D0%B7%D1%8B%D1%87%D0%BD%D1%8B%D0%B5-%D0%B8%D1%81%D1%82%D0%BE%D1%87%D0%BD%D0%B8%D0%BA%D0%B8-%D0%BF%D1%80%D0%B5%D0%B4%D1%88%D0%B5%D1%81%D1%82%D0%B2%D0%BE%D0%B2%D0%B0%D0%BB%D0%B8-%D0%BE%D1%81%D1%82%D0%B0%D0%BB%D1%8C%D0%BD%D1%8B%D0%BC
%\DeclareSourcemap{
%	\maps[datatype=bibtex]{
%		\map{
%			\step[fieldset=langid, fieldvalue={tempruorder}]
%		}
%		\map[overwrite]{
%			\step[fieldsource=langid, match=russian, final]
%			\step[fieldsource=presort, 
%			match=\regexp{(.+)}, 
%			replace=\regexp{aa$1}]
%		}
%		\map{
%			\step[fieldsource=langid, match=russian, final]
%			\step[fieldset=presort, fieldvalue={az}]
%		}
%		\map[overwrite]{
%			\step[fieldsource=langid, notmatch=russian, final]
%			\step[fieldsource=presort, 
%			match=\regexp{(.+)}, 
%			replace=\regexp{za$1}]
%		}
%		\map{
%			\step[fieldsource=langid, notmatch=russian, final]
%			\step[fieldset=presort, fieldvalue={zz}]
%		}
%		\map{
%			\step[fieldsource=langid, match={tempruorder}, final]
%			\step[fieldset=langid, null]
%		}
%	}
%}
%Сортировка списка литературы (конец)

%%% Создание команд для вывода списка литературы %%%
\newcommand*{\insertbibliofull}{
	\printbibliography[keyword=bibliofull,section=0,title=\bibtitlefull]
	\ifnumequal{\value{draft}}{0}{
		\printbibliography[heading=nobibheading,env=counter,keyword=bibliofull,section=0]
	}{}
}
\newcommand*{\insertbiblioauthor}{
	\printbibliography[heading=pubgroup, section=0, keyword=biblioauthor, title=\bibtitleauthor]
}
\newcommand*{\insertbiblioauthorimportant}{
	\printbibliography[heading=pubgroup, section=2, keyword=biblioauthor, title=\bibtitleauthorimportant]
}

% Вариант вывода печатных работ автора, с группировкой по типу источника.
% Порядок команд `\printbibliography` должен соответствовать порядку в файле common/characteristic.tex
\newcommand*{\insertbiblioauthorgrouped}{
	\section*{\bibtitleauthor}
	\ifsynopsis
	\printbibliography[heading=pubsubgroup, section=0, keyword=biblioauthorvak,    title=\bibtitleauthorvak,resetnumbers=true] % Работы автора из списка ВАК (сброс нумерации)
	\else
	\printbibliography[heading=pubsubgroup, section=0, keyword=biblioauthorvak,    title=\bibtitleauthorvak,resetnumbers=false] % Работы автора из списка ВАК (сквозная нумерация)
	\fi
	\printbibliography[heading=pubsubgroup, section=0, keyword=biblioauthorwos,    title=\bibtitleauthorwos,resetnumbers=false]% Работы автора, индексируемые Web of Science
	\printbibliography[heading=pubsubgroup, section=0, keyword=biblioauthorscopus, title=\bibtitleauthorscopus,resetnumbers=false]% Работы автора, индексируемые Scopus
	\printbibliography[heading=pubsubgroup, section=0, keyword=biblioauthorconf,   title=\bibtitleauthorconf,resetnumbers=false]% Тезисы конференций
	\printbibliography[heading=pubsubgroup, section=0, keyword=biblioauthorother,  title=\bibtitleauthorother,resetnumbers=false]% Прочие работы автора
}

\newcommand*{\insertbiblioexternal}{
	\printbibliography[heading=pubgroup,    section=0, keyword=biblioexternal,     title=\bibtitlefull]
}
     % Реализация пакетом biblatex через движок biber

%%% Управление компиляцией отдельных частей документа %%%
% Необходимо сначала иметь полностью скомпилированный документ, чтобы все
% промежуточные файлы были в наличии
% Затем, для вывода отдельных частей можно воспользоваться командой \includeonly
% Ниже примеры использования команды:
%
%\includeonly{text/part5_exercises}
%
% Если все команды закомментированы, то документ будет выведен в PDF файл полностью

