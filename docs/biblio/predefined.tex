%%% Реализация библиографии встроенными средствами посредством движка bibtex8 %%%

%%% Пакеты %%%
\usepackage{cite}                                   % Красивые ссылки на литературу


%%% Стили %%%
\bibliographystyle{BibTeX-Styles/utf8gost71u}    % Оформляем библиографию по ГОСТ 7.1 (ГОСТ Р 7.0.11-2011, 5.6.7)

\makeatletter
\renewcommand{\@biblabel}[1]{#1.}   % Заменяем библиографию с квадратных скобок на точку
\makeatother
%% Управление отступами между записями
%% требует etoolbox
%% http://tex.stackexchange.com/a/105642
%\patchcmd\thebibliography
% {\labelsep}
% {\labelsep\itemsep=5pt\parsep=0pt\relax}
% {}
% {\typeout{Couldn't patch the command}}

%%% Список литературы с красной строки (без висячего отступа) %%%
%\patchcmd{\thebibliography} %может потребовать включения пакета etoolbox
%  {\advance\leftmargin\labelsep}
%  {\leftmargin=0pt%
%   \setlength{\labelsep}{\widthof{\ }}% Управляет длиной отступа после точки
%   \itemindent=\parindent%
%   \addtolength{\itemindent}{\labelwidth}% Сдвигаем правее на величину номера с точкой
%   \advance\itemindent\labelsep%
%  }
%  {}{}

%%% Цитирование %%%
\renewcommand\citepunct{;\penalty\citepunctpenalty%
    \hskip.13emplus.1emminus.1em\relax}                % Разделение ; при перечислении ссылок (ГОСТ Р 7.0.5-2008)

\newcommand*{\autocite}[1]{}  % Чтобы примеры цитирования, рассчитанные на biblatex, не вызывали ошибок при компиляции в bibtex

%%% Создание команд для вывода списка литературы %%%
\newcommand*{\insertbibliofull}{
\bibliography{biblio/external,biblio/author}         % Подключаем BibTeX-базы % После запятых не должно быть лишних пробелов — он "думает", что это тоже имя пути
}

\newcommand*{\insertbiblioauthor}{
\bibliography{biblio/author}         % Подключаем BibTeX-базы % После запятых не должно быть лишних пробелов — он "думает", что это тоже имя пути
}

\newcommand*{\insertbiblioexternal}{
\bibliography{biblio/external}         % Подключаем BibTeX-базы
}


%% Счётчик использованных ссылок на литературу, обрабатывающий с учётом неоднократных ссылок
%% Требуется дважды компилировать, поскольку ему нужно считать актуальный внешний файл со списком литературы
\newtotcounter{citenum}
\def\oldcite{}
\let\oldcite=\bibcite
\def\bibcite{\stepcounter{citenum}\oldcite}
