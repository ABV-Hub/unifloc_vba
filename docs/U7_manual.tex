\documentclass[]{scrreprt}
%\usepackage[utf8]{inputenc}
\usepackage{graphicx}
\usepackage{cite}
\usepackage{polyglossia}

\setdefaultlanguage[spelling=modern]{russian}
\setotherlanguage{english}
\setmainfont{Times New Roman}
\setsansfont{Arial}
\setmonofont{Courier New}
\newfontfamily\cyrillicfonttt[Script=Cyrillic]{Courier New}


\usepackage{amsmath, amsfonts, amssymb, mathtools}
%\usepackage[russian]{babel} 

\usepackage{minted}
\usepackage{tcolorbox}

\usepackage{pgfplotstable}
\usepackage{pgfplots}

\tcbuselibrary{breakable,skins,minted}

\renewcommand{\listingscaption}{Листинг}

\usepackage{tikz}
\usepackage{pgfplots}
\pgfplotsset{compat=1.12}

\usepackage[hidelinks]{hyperref}
%\usepackage[colorlinks=true, urlcolor=blue, pdfborder={0 0 0}]{hyperref}


\usemintedstyle{vs}

\graphicspath{{pics/}}


\bibliographystyle{utf8gost705u}

\begin{document}

% version definition
\newcommand{\unf}{Unifloc 7.3 VBA }

\newcommand{\putlisting}[1]{
	\tcbinputlisting{
		listing file=#1,
		minted language=vb.net,
		minted options={breaklines,fontsize=\small},% <-- put other minted options inside the brackets
		breakable,enhanced,% <-- put other tcolorbox options here
		listing only
		}
}


%\begin{titlepage} 
%	\textbf{\Huge Инженерные расчёты при механизированной добыче нефти} 
%\end{titlepage} 

\title{\unf Руководство пользователя}
\author{Хабибуллин Ринат}

\maketitle

\tableofcontents

\chapter{Введение}
Новая версия

Настоящее описание посвящено использованию расчётных модулей \unf реализованных в Excel VBA. Расчётные модули предназначены для изучения математических моделей систем нефтедобычи в рамках учебных курсов "Механизированная добыча нефти", "Инженерные расчёты в добыче нефти" и других.

Расчётные модули охватывают основные элементы математических моделей систем нефтедобычи - модель физико-химических свойств пластовых флюидов, модели многофазного потока в трубах, в пласте, задачи узлового анализа, модели скважинного оборудования в частности УЭЦН.  

Для использования расчётных модулей требуются навыки уверенного пользователя MS Excel, желательно знание основ программирования и основ теории добычи нефти. 

Алгоритмы реализованные в расчётных модулей не претендуют на полноту и достоверность и предназначены в первую очередь для изучения основ моделирования систем нефтедобычи и проведения простых расчётов (создания простых расчётных модулей). Руководство пользователя также не претендует на полноту описания системы. Приводится как есть. Более надёжным способом получения достоверной информации о работе макросов Унифлок 7 является изучение непосредственно расчётного кода в редакторе VBE.

По всем вопросам можно обращаться к автору расчётных модулей - Хабибуллину Ринату (khabibullin.ra@gubkin.ru)  

Настоящее описание посвящено версии \unf  расчётных модулей 

\chapter{Макросы VBA для проведения расчётов}

Расчеты в рамках курса выполняются с использованием макросов написанных на языке программирования Visual Basic for Application (VBA) в среде Excel.

Для использования макросов не требуется программировать, достаточно уметь вызывать необходимые макросы из Excel. Тем не менее макросы могут быть использованы для написания собственных подпрограмм или модифицированы для достижения необходимых целей. Владение навыками программирования и изучения исходного кода макросов может оказаться чрезвычайно полезным. 

Исходный код расчётных модулей находится в отдельном файле - надстройке Excel файле с расширением.xlam. Для использования макросов данная надстройка должна быть установлена на компьютере, на котором проводятся расчёты. Подробное описание процедуры установки надстройки можно найти на сайте microsoft по ключевым словам \href{https://support.office.com/ru-ru/article/%D0%94%D0%BE%D0%B1%D0%B0%D0%B2%D0%BB%D0%B5%D0%BD%D0%B8%D0%B5-%D0%B8-%D1%83%D0%B4%D0%B0%D0%BB%D0%B5%D0%BD%D0%B8%D0%B5-%D0%BD%D0%B0%D0%B4%D1%81%D1%82%D1%80%D0%BE%D0%B5%D0%BA-%D0%B2-excel-0af570c4-5cf3-4fa9-9b88-403625a0b460}{"добавление и удаление надстроек в Excel"}.

Для активации надстройки 
\begin{enumerate}
	\item На вкладке Файл выберите команду Параметры, а затем — категорию Надстройки.
	\item В поле Управление выберите пункт Надстройки Excel, а затем нажмите кнопку Перейти. Откроется диалоговое окно Надстройки.
	\item Чтобы установить и активировать надстройку Унифлок 7.1, нажмите кнопку Обзор (в диалоговом окне Надстройки), найдите надстройку, а затем нажмите кнопку ОК.
	\item Надстройка появится в списке надстроек. Галочка активации надстройки должна быть установлена
\end{enumerate}	
После установки и активации надстройки, встроенными в нее макросами можно будет пользоваться в любой книге Excel на данным компьюетере. При переносе расчётный файлов на другой комппьютер для сохранения их работоспособности должна быть передана и установлена и надстройка. 


\section{Запуск VBA}

Чтобы получить доступ к макросам в текущей версии расчётного модуля для выполнения упражнений необходимо:
\begin{itemize}
	\item Запустить Excel запустив рабочую книгу для выполнения упражнений
	\item Нажать комбинацию клавиш <Alt-F11>
	\item Откроется новое окно c редактором макросов VBA (Рис. \ref{ris:VBA_overview}). Иногда в литературе окно редактирования макросов обозначают как VBE (Visual Basic Enviroment)
	\item Окне VBE можно изучить структуру проекта (набора макросов и других элементов). Раздел со структурой проекта можно открыть из меню <Вид – Обозреватель проекта>. Макросы располагаются в ветках «модули» и «модули классов»
	 
\end{itemize}

\begin{figure}[h]
	\center{\includegraphics[width=1\linewidth]{VBA_overview}}
	\caption{Схема зависимости свойств флюидов}
	\label{ris:VBA_overview}
\end{figure}

\section{Ключевые особенности VBA и соглашения, используемые в макросах}
Строки, начинающиеся со знака ‘ являются комментариями. В VBE они выделяются зеленым цветом. На исполнение макроса не влияют.

Для многих макросов не обязательно задавать все параметры. Некоторые значения параметров могут не задаваться – тогда будут использованы значения параметров, принятые по умолчанию. Параметры, допускающие задание по умолчанию помечены в исходном коде ключевым словом \mintinline{vb.net}{Optional}.

\section{Обозначение параметров}
При создании макросов в основном использовались международные обозначения переменных принятые в монографиях общества инженеров нефтяников SPE.

\chapter{Функции модуля «u7\_Excel\_functions»}
\section{Расчет физико-химических свойств флюидов (PVT)}
Для расчёта физико-химических свойств пластовых флюидов используется модель нелетучей нефти. Для всех функций, реализующих расчёт с учётом PVT свойств необходимо задавать одинаковый полный набор параметров, описывающих нефть, газ и воду.  При этом для некоторых частных функций не все параметры будут влиять на результат расчёта, тем не менее эти параметры необходимо задавать. Это сделано для унификации методик расчёта – при любом вызове функции проводится расчёт всех свойств модели нелетучей нефти, но возвращаются только необходимые данные. Это обстоятельности может замедлить расчёты с использованием функций Excel.
 
\subsection{Обозначения PVT параметров}
Типовой набор параметров приведен ниже:

\putlisting{listings/uf7_vars.txt}


\begin{itemize}
	
\item	$\gamma_g$  - \mintinline{vb.net}{gamma_gas} - удельная плотность газа, по воздуху. Стандартное обозначение переменной \mintinline{vb.net}{gamma_gas}. Безразмерная величина. Следует обратить внимание, что удельная плотность газа по воздуху не совпадает с плотностью воздуха в г/см3, поскольку плотность воздуха при стандартных условиях \mintinline[breaklines]{vb.net}{Const const_rho_air = 1.205} при температуре 20 °С и давлении 101325 Па для сухого воздуха. По умолчанию задается значение \mintinline[breaklines]{vb.net}{const_gg_default = 0.6}

\item $\gamma_o$  - \mintinline{vb.net}{gamma_oil} - удельная плотность нефти, по воде. Стандартное обозначение переменной \mintinline{vb.net}{gamma_oil}. Безразмерная величина, но по значению совпадает с плотность в г/см3. По умолчанию задаётся значение \mintinline{vb.net}{const_go_default = 0.86}

\item $\gamma_w$  - \mintinline{vb.net}{gamma_wat}- удельная плотность воды, по воде. Стандартное обозначение переменной \mintinline{vb.net}{gamma_wat}. Безразмерная величина, но по значению совпадает с плотность в г/см3. По умолчанию задаётся значение \mintinline{vb.net}{const_gw_default = 1} Плотность воды может отличаться от задаваемой по умолчанию, например для воды с большой минерализацией.  

\item $R_{sb}$- газосодержание при давлении насыщения, м3/м3. Стандартное обозначение в коде \mintinline{vb.net}{Rsb_m3m3}. Значение по умолчанию \mintinline{vb.net}{const_Rsb_default = 100}

\item $R_p$-  замерной газовый фактор, м3/м3. Стандартное обозначение в коде \mintinline{vb.net}{Rp_m3m3}. Калибровочный параметр. По умолчанию используется значение равное газосодержанию при давлении насыщения. Если задаётся значение меньшее чем газосодержание при давлении насыщения, то последнее принимается равным газовому фактору (приоритет у газового фактора, потому что как правило это замерное значение в отличии от газосодержания определяемого по результатам лабораторных исследований проб нефти).

\item $P_b$ - давление насыщения, атм. Стандартное обозначение в коде \mintinline{vb.net}{Pb_atm}. Калибровочный параметр. По умолчанию не задаётся, рассчитывается по корреляции. Если задан, то все расчёты по корреляциям корректируются с учётом заданного параметра. При задании давления насыщения обязательно должна быть задана температура пласта – температура при которой было определено давление насыщения. 

\item $T_{res}$- пластовая температура, С. Стандартное обозначение в коде \mintinline{vb.net}{Tres_C}. Учитывается при расчёте давления насыщения. \mintinline{vb.net}{const_Tres_default = 90}

\item $B_{ob}$ - объёмный коэффициент нефти, м3/м3. Стандартное обозначение в коде \mintinline{vb.net}{Bob_m3m3}. Калибровочный параметр. По умолчанию рассчитывается по корреляции. Если задан, то все расчёты по корреляциям корректируются с учётом заданного параметра.

\item $\mu_{ob}$ - вязкость нефти при давлении насыщения, сП. Стандартное обозначение \mintinline{vb.net}{Muob_cP}. Калибровочный параметр. По умолчанию рассчитывается по корреляции. Если задан, то все расчёты по корреляциям корректируются с учётом заданного параметра.

\item PVTcorr - номер набора PVT корреляций используемых для расчёта

\begin{itemize}	
	\item 	StandingBased = 0 - на основе корреляции Стендинга
	\item 	McCainBased = 1 - на основе корреляции Маккейна
	\item 	StraigthLine = 2 - на основе упрощенных зависимостей
\end{itemize}

\item $K_s$ – коэффициент сепарации газа. Определяет изменение свойств флюида после отделения части газа из потока в результате сепарации при определённых давлении и температуре. По умолчанию предполагается, что сепарации нет $K_s$=0

\end{itemize}

\subsection{Стандартные условия} 
Многие параметры нефти, газа и воды существенно зависят от давления и температуры. Например объем занимаемый определённым количеством газа примерно в два раза снизится при повышении давления в два раза. 

Поэтому для удобства фиксации и сравнения параметров они часто приводятся к \href{https://ru.wikipedia.org/wiki/%D0%A1%D1%82%D0%B0%D0%BD%D0%B4%D0%B0%D1%80%D1%82%D0%BD%D1%8B%D0%B5_%D1%83%D1%81%D0%BB%D0%BE%D0%B2%D0%B8%D1%8F}{стандартным или нормальным условиям} - определённым давлениям и температуре. 
	
Принятые в разных дисциплинах и разных организациях точные значения давления и температуры в стандартных условиях могут различаться (смотри например \url{https://en.wikipedia.org/wiki/Standard_conditions_for_temperature_and_pressure}), поэтому указание значений физических величин без уточнения условий, в которых они приводятся, может приводить к ошибкам. Наряду с термином «стандартные условия» применяется термин «нормальные условия». «Нормальные условия» обычно отличаются от «стандартных» тем, что под нормальным давлением принимается давление равное 101 325 Па = 1 атм = 760 мм рт. ст.

Обычно в монографиях SPE принято, что стандартное давление для газов, жидкостей и твёрдых тел, равное $10^5$ Па (100 кПа, 1 бар); стандартная температура для газов, равная 15.6 °С соответствующая 60 °F. 

В Российском ГОСТ 2939-63  принято, что стандартное давление для газов, жидкостей и твёрдых тел, равное $10.13^5$ Па (101325 Па, 1 атм); стандартная температура для газов, равная 20 °С соответствующая 68 °F. 

В \unf приняты следующие значения стандартных условий

\begin{minted}{vb.net}
	Public Const const_Psc_atma As Double = 1
	Public Const const_Tsc_C = 20
	Public Const const_convert_atma_Pa = 101325
\end{minted}


\subsection{PVT\_Pb\_atma – давление насыщения}
Функция рассчитывает давление насыщения по известным данным газосодержания при давлении насыщения, $\gamma_g, \gamma_o, T_r$.

При проведении расчётов с использованием значения давления насыщения, следует помнить, что давление насыщения является функцией температуры. В частности при калибровки результатов расчётов на известное значение давления насыщения $P_b$ следует указывать значение пластовой температуры $T_r$ при котором давление насыщения было получено. 

В наборе корреляций на основе корреляции Стендинга расчет давления насыщения проводится по корреляции Стендинга \cite{Yukos_PVT_2002}

\putlisting{listings/PVT_Pb_atma.txt}

Примеры расчёта с использованием функции \mintinline{vb.net}{PVT_Pb_atma} для различных наборов PVT корреляций приведён на рисунке ниже. Видно, что результаты расчетов по различным корреляциях дают качественно схожие результаты, но не совпадают друг с другом.  Отличия, по всей видимости,  обусловные применением различных наборов исходных данных использовавшихся авторами. Поэтому при проведении расчетов для конкретного месторождения актуальной является задача выбора адекватного набора корреляций. Макросы \unf позволяют провести расчет с использованием различных подходов, но при этом выбор корреляции остается за пользователем. 

\begin{tikzpicture}[scale=0.8]
\begin{axis}[
xlabel=$R_{sb} \;  m^3/m^3$,
ylabel=$P_b\; atma$,
legend pos=north west,
title=Standing]
\addplot table [y=T20, x=Rs]{data/Pb_T_data.txt};
\addlegendentry{$T = 20$ С}
\addplot table [y=T60, x=Rs]{data/Pb_T_data.txt};
\addlegendentry{$T = 60$ С}
\addplot table [y=T100, x=Rs]{data/Pb_T_data.txt};
\addlegendentry{$T = 100$ С}
\addplot table [y=T140, x=Rs]{data/Pb_T_data.txt};
\addlegendentry{$T = 140$ С}
\end{axis}
\end{tikzpicture}
\begin{tikzpicture}[scale=0.8]
\begin{axis}[
xlabel=$R_{sb} \;  m^3/m^3$,
ylabel=$P_b\; atma$,
legend pos=north west,
title = McCain]
\addplot table [y=T20, x=Rs]{data/Pb_T_data1.txt};
\addlegendentry{$T = 20$ С}
\addplot table [y=T60, x=Rs]{data/Pb_T_data1.txt};
\addlegendentry{$T = 60$ С}
\addplot table [y=T100, x=Rs]{data/Pb_T_data1.txt};
\addlegendentry{$T = 100$ С}
\addplot table [y=T140, x=Rs]{data/Pb_T_data1.txt};
\addlegendentry{$T = 140$ С}
\end{axis}
\end{tikzpicture}


При проведении расчётов с использованием набора корреляций на основе корреляций МакКейна следует учитывать, что они работают только для температур более 18 градусов Цельсия. При более низких значениях температуры расчёт будет проводить для 18 градусов Цельсия. 

\subsection{PVT\_Rs\_m3m3 – газосодержание}

Газосодержание это отношения объёма газа растворенного в нефти к объёму нефти приведённые к стандартным условиям. 

$$R_s = \frac{(V_g)_{sc}}{(V_o)_{sc}}$$

Газосодержание является одним из ключевых свойств нефти при расчётах производительности скважин и работы скважинного оборудования. Динамика изменения газосодержания во многом определяет количество свободного газа в потоке и должна учитываться при проведении расчётов. 

При задании PVT свойств нефти часто используют значение газосодержания при давлении насыщения $r_{sb}$ - определяющее объем газа растворенного в нефти в пластовых условиях. В модели флюида \unf газосодержание при давлении насыщения является исходным параметров нефти и должно быть обязательно задано. 

Следует отличать газосодержание в нефти при давлении насыщения $R_sb$ и газовый фактор $R_p$.

$$R_p = \frac{(Q_g)_{sc}}{(Q_o)_{sc}}$$

Газовый фактор $R_{p}$  в отличии от газосодержания $R_{sb}$  является, вообще говоря, параметром скважины - показывает отношение объёма добытого газа из скважины к объёму добытой нефти приведённые к стандартным условиям. Газосодержание же является свойством нефти - показывает сколько газа растворено в нефти. Если газ добываемой из скважины это газ который выделился из нефти в процессе подъёма, что характерно для недонасыщенных нефтей, то значения газового фактора и газосодержания будут совпадать. Если газ поступает в скважину не непосредственно из добываемой нефти, а например фильтруется из газовой шапки или поступает через негерметичность ствола скважины - то в такой скважине газовый фактор может значительно превышать значение газосодержания. Такая ситуация может быть смоделирована в \unf. Для этого необходимо наряду с газосодержанием при давлении насыщения $R_{sb}$ задать значение газового фактора $R_p$. В этом случае газосодержание при давлении насыщения $R_{sb}$  будет определять динамику выделения попутного газа из нефти при снижении давления, а газовый фактор $R_p$ определять общее количество газа в потоке. 

При определённых условиях газовый фактор может быть меньше газосодержания. Это происходит, когда газ выделяется в призабойной зоне и скапливается в ней не поступая в скважину вместе с нефтью. Но такие условия возникают достаточно редко, существуют на скважине ограниченное время и представляют интерес больше для разработчиков нежели чем для технологов. С точки зрения анализа работы скважины и скважинного оборудования можно считать, что значение газового фактора не может быть меньше газосодержания при давлении насыщения. Такой предположение реализовано в \unf. При этом значение газового фактора технически легче измерить чем газосодержание - поэтому при противоречии значений газового фактора и газосодержания при давлении насыщения приоритет отдаётеся газовому фактору. 


\putlisting{listings/PVT_Rs_m3m3.txt}

Примеры расчёта с использованием функции \mintinline{vb.net}{PVT_Rs_m3m3} для различных наборов PVT корреляций приведён на рисунке ниже.

\newcommand{\RsDataFile}{data/Rs_P_data.txt}
\begin{tikzpicture}[scale=0.8]
\begin{axis}[
xlabel=$R_{sb} \;  m^3/m^3$,
ylabel=$P_b\; atma$,
legend pos=south east,
title=Standing]
\addplot table [y=T_0_20, x=P]{\RsDataFile};
\addlegendentry{$T = 20$ С}
\addplot table [y=T_0_60, x=P]{\RsDataFile};
\addlegendentry{$T = 60$ С}
\addplot table [y=T_0_100, x=P]{\RsDataFile};
\addlegendentry{$T = 100$ С}
\addplot table [y=T_0_140, x=P]{\RsDataFile};
\addlegendentry{$T = 140$ С}
\end{axis}
\end{tikzpicture}
\begin{tikzpicture}[scale=0.8]
\begin{axis}[
xlabel=$R_{sb} \;  m^3/m^3$,
ylabel=$P_b\; atma$,
legend pos=south east,
title = McCain]
\addplot table [y=T_1_20, x=P]{\RsDataFile};
\addlegendentry{$T = 20$ С}
\addplot table [y=T_1_60, x=P]{\RsDataFile};
\addlegendentry{$T = 60$ С}
\addplot table [y=T_1_100, x=P]{\RsDataFile};
\addlegendentry{$T = 100$ С}
\addplot table [y=T_1_140, x=P]{\RsDataFile};
\addlegendentry{$T = 140$ С}
\end{axis}
\end{tikzpicture}


\subsection{PVT\_Bo\_m3m3 – объёмный коэффициент нефти}

Функция рассчитывает объёмный коэффициент нефти для произвольных термобарических условий. 
Объёмный коэффициент нефти определяется как отношение объёма занимаемого нефтью в пластовых условиях к объёму занимаемому нефтью при стандартных условиях. 

$$B_o = \frac{(V_o)_{rc}}{(V_o)_{sc}}$$

Нефть в пласте занимает больший объем чем на поверхности за счёт растворенного в ней газа. Соответственно объёмный коэффициент нефти обычно имеет значение больше 1 при давлениях больше чем стандартное.

Для калибровки значения объёмного коэффициента можно использовать значение объёмного коэффициента нефти при давлении насыщения $B_{ob}$. 

Следует отметить, что вообще говоря значение объёмного коэффициента нефти при давлении насыщения не является значением при пластовых условиях (при давлении выше давления насыщения играет роль сжимаемость нефти), однако при анализе производительности скважины и скважинного оборудования можно условно считать, что значение объёмного коэффициента при давлении насыщения соответствует значению  объёмного коэффициента в пластовых условиях.  

\putlisting{listings/PVT_Bo_m3m3.txt}

Примеры расчёта с использованием функции \mintinline{vb.net}{PVT_Bo_m3m3} для различных наборов PVT корреляций приведён на рисунке ниже.

Объёмный коэффициент нефти хорошо коррелирует со значением газосодержания. Поэтому различный вид кривых на рисунке ниже связан с первую очередь с различным газосодержанием при проведении расчётов.

\newcommand{\BoDataFile}{data/Bo_P_data.txt}
\begin{tikzpicture}[scale=0.8]
\begin{axis}[
xlabel=$P\; atma$,
ylabel=$B_o\;  m^3/m^3$,
legend pos=south east,
title=Standing]
\addplot table [y=T_0_20, x=P]{\BoDataFile};
\addlegendentry{$T = 20$ С}
\addplot table [y=T_0_60, x=P]{\BoDataFile};
\addlegendentry{$T = 60$ С}
\addplot table [y=T_0_100, x=P]{\BoDataFile};
\addlegendentry{$T = 100$ С}
\addplot table [y=T_0_140, x=P]{\BoDataFile};
\addlegendentry{$T = 140$ С}
\end{axis}
\end{tikzpicture}
\begin{tikzpicture}[scale=0.8]
\begin{axis}[
xlabel=$P\; atma$,
ylabel=$B_o\;  m^3/m^3$,
legend pos=south east,
title = McCain]
\addplot table [y=T_1_20, x=P]{\BoDataFile};
\addlegendentry{$T = 20$ С}
\addplot table [y=T_1_60, x=P]{\BoDataFile};
\addlegendentry{$T = 60$ С}
\addplot table [y=T_1_100, x=P]{\BoDataFile};
\addlegendentry{$T = 100$ С}
\addplot table [y=T_1_140, x=P]{\BoDataFile};
\addlegendentry{$T = 140$ С}
\end{axis}
\end{tikzpicture}

\subsection{PVT\_Bg\_m3m3 – объёмный коэффициент газа}
Функция рассчитывает объёмный коэффициент нефтяного газа для произвольных термобарических условий. 

Объёмный коэффициент газа определяется как отношение объема занимаемого газом для произвольных термобарических условий (при определенном давлении и температуре) к объёму занимаемому газом при стандартных условиях. 

$$B_g = \frac{V_g(P,T)}{(V_g)_{sc}}$$

Значение объемного коэффиента газа может быть определено исходя из уравнения состояния газа

$$ PV = z \nu RT  $$

откуда можно получить 

$$ B_g = z \frac{P_{sc}}{P} \frac{T}{T_{sc}} $$

где $P_{sc}, T_{sc}$ давление (атм) и температура (К) при стандартных условиях, $P,T$ давление (атм) и температура (K) при расчетных условиях, $z$ коэффициент сверхсжимаемости газа, который вообще говоря зависит от давления и температуры $z = z(P,T)$. 

\putlisting{listings/PVT_Bg_m3m3.txt}

\newcommand{\DataFile}{data/Bg_P_data.txt}
\begin{tikzpicture}[scale=0.8]
\begin{axis}[
ymode=log, 
xlabel=$P\; atma$,
ylabel=$B_g\\;  m^3/m^3$,
legend pos=north east,
title=Standing]
\addplot table [y=T_0_20, x=P]{\DataFile};
\addlegendentry{$T = 20$ С}
\addplot table [y=T_0_60, x=P]{\DataFile};
\addlegendentry{$T = 60$ С}
\addplot table [y=T_0_100, x=P]{\DataFile};
\addlegendentry{$T = 100$ С}
\addplot table [y=T_0_140, x=P]{\DataFile};
\addlegendentry{$T = 140$ С}
\end{axis}
\end{tikzpicture}
\begin{tikzpicture}[scale=0.8]
\begin{axis}[
ymode=log, 
xlabel=$P\; atma$,
ylabel=$B_g\;  m^3/m^3$,
legend pos=north east,
title = McCain]
\addplot table [y=T_1_20, x=P]{\DataFile};
\addlegendentry{$T = 20$ С}
\addplot table [y=T_1_60, x=P]{\DataFile};
\addlegendentry{$T = 60$ С}
\addplot table [y=T_1_100, x=P]{\DataFile};
\addlegendentry{$T = 100$ С}
\addplot table [y=T_1_140, x=P]{\DataFile};
\addlegendentry{$T = 140$ С}
\end{axis}
\end{tikzpicture}


\subsection{PVT\_Muo\_cP – вязкость нефти}
Функция рассчитывает вязкость нефти при заданных термобарических условиях по корреляции. Расчёт может быть откалиброван на известное значение вязкости нефти при давлении равном давлению насыщения и при пластовой температуре за счёт задания калибровочного параметра \mintinline{vb.net}{Muob_cP}. При калибровке динамика изменения будет соответствовать расчету по корреляции, но значения будут масштабированы таким образом, чтобы при давлении насыщения удовлетворить калибровочному параметру.

При расчёте следует обратить внимание, что значение вязкости коррелирует со значением плотности нефти. Как правило вязкость тяжёлых нефтей выше чем для легких.

При расчёте с использованием набора корреляций на основе корреляции Стендинга - вязкость как дегазированной нефти и нефти с учетом растворенного газа рассчитывается по корреляции Беггса Робинсона \cite{Yukos_PVT_2002}. 
Корреляции для расчета вязкости разгазированной и газонасыщенной нефти, разработанные Beggs \& Robinson, основаны на 2000 замерах 600 различных нефтей.
Диапазоны значений основных свойств, использованных для разработки данной корреляции, приведены в таблице ниже.
\begin{center}
	\begin{tabular}{ccc}
		давление, atma & \textbf{8.96…483.} \\
		температура, °C & \textbf{37…127}  \\
		газосодержание, $R_s \; m^3 /m^3$ & \textbf{3.6…254}\\
		относительная плотность нефти по воде,, $\gamma_o$ & \textbf{0.725…0.956} \\
	\end{tabular}
\end{center}
   
\putlisting{listings/PVT_Muo_cP.txt}

\newcommand{\MuDataFile}{data/Muo_P_data.txt}
\begin{tikzpicture}[scale=0.8]
\begin{axis}[
xlabel=$P\; atma$,
ylabel=$\mu_o\; cP$,
legend pos=north east,
title=Standing]
\addplot table [y=T_0_20, x=P]{\MuDataFile};
\addlegendentry{$T = 20$ С}
\addplot table [y=T_0_60, x=P]{\MuDataFile};
\addlegendentry{$T = 60$ С}
\addplot table [y=T_0_100, x=P]{\MuDataFile};
\addlegendentry{$T = 100$ С}
\addplot table [y=T_0_140, x=P]{\MuDataFile};
\addlegendentry{$T = 140$ С}
\end{axis}
\end{tikzpicture}
\begin{tikzpicture}[scale=0.8]
\begin{axis}[
xlabel=$P\; atma$,
ylabel=$\mu_o\; cP$,
legend pos=north east,
title = McCain]
\addplot table [y=T_1_20, x=P]{\MuDataFile};
\addlegendentry{$T = 20$ С}
\addplot table [y=T_1_60, x=P]{\MuDataFile};
\addlegendentry{$T = 60$ С}
\addplot table [y=T_1_100, x=P]{\MuDataFile};
\addlegendentry{$T = 100$ С}
\addplot table [y=T_1_140, x=P]{\MuDataFile};
\addlegendentry{$T = 140$ С}
\end{axis}
\end{tikzpicture}

\subsection{PVT\_Mug\_cP – вязкость газа}

Функция рассчитывает вязкость газа при заданных термобарических условиях. Результат расчета в сП.  Используется подход предложенный Lee  \cite{Lee_1966}, который хорошо подходит для большинства натуральных газов. 
В отличии от нефти и других жидкостей вязкость газа, как правило, значительно ниже, что определяет высокую подвижность газа. 
Более подробное описание методов расчета вязкости газа можно найти на странице  \href{http://petrowiki.org/Gas_viscosity}{http://petrowiki.org/gas\_viscosity}


\putlisting{listings/PVT_Mug_cP.txt}

\newcommand{\MugDataFile}{data/Mug_P_data.txt}
\begin{tikzpicture}[scale=0.8]
\begin{axis}[
xlabel=$P\; atma$,
ylabel=$\mu_g\; cP$,
legend pos=north west,
title=Standing]
\addplot table [y=T_0_20, x=P]{\MugDataFile};
\addlegendentry{$T = 20$ С}
\addplot table [y=T_0_60, x=P]{\MugDataFile};
\addlegendentry{$T = 60$ С}
\addplot table [y=T_0_100, x=P]{\MugDataFile};
\addlegendentry{$T = 100$ С}
\addplot table [y=T_0_140, x=P]{\MugDataFile};
\addlegendentry{$T = 140$ С}
\end{axis}
\end{tikzpicture}
\begin{tikzpicture}[scale=0.8]
\begin{axis}[
xlabel=$P\; atma$,
ylabel=$\mu_g\; cP$,
legend pos=north west,
title = McCain]
\addplot table [y=T_1_20, x=P]{\MugDataFile};
\addlegendentry{$T = 20$ С}
\addplot table [y=T_1_60, x=P]{\MugDataFile};
\addlegendentry{$T = 60$ С}
\addplot table [y=T_1_100, x=P]{\MugDataFile};
\addlegendentry{$T = 100$ С}
\addplot table [y=T_1_140, x=P]{\MugDataFile};
\addlegendentry{$T = 140$ С}
\end{axis}
\end{tikzpicture}

\subsection{PVT\_Muw\_cP – вязкость воды}

Функция рассчитывает вязкость воды при заданных термобарических условиях. Результат расчета выдается в сП. 
Вязкость воды зависит от давления, температуры и наличия растворенных примесей. В общем вязкость аоды растет при росте давления, снижении температуры, повышении солености. 
Растворение газа почти не влияет на вязкость воды и в расчетах не учитывается. 
Расчет проводится по корреляции McCain \cite{McCain_1991}

Более подробное описание методов расчета вязкости газа можно найти на странице  \href{http://petrowiki.org/Produced_water_properties}{http://petrowiki.org/Produced\_water\_properties}


\putlisting{listings/PVT_Muw_cP.txt}


Следует отметить, что вязкость воды достаточно сильно зависит от температуры, в то время как зависимость от давления менее выражена.

\newcommand{\MuwDataFile}{data/Muw_P_data.txt}
\begin{tikzpicture}[scale=0.8]
\begin{axis}[
xlabel=$P\; atma$,
ylabel=$\mu_w\; cP$,
legend pos=north west,
title=Standing]
\addplot table [y=T_0_20, x=P]{\MuwDataFile};
\addlegendentry{$T = 20$ С}
\addplot table [y=T_0_60, x=P]{\MuwDataFile};
\addlegendentry{$T = 60$ С}
\addplot table [y=T_0_100, x=P]{\MuwDataFile};
\addlegendentry{$T = 100$ С}
\addplot table [y=T_0_140, x=P]{\MuwDataFile};
\addlegendentry{$T = 140$ С}
\end{axis}
\end{tikzpicture}
\begin{tikzpicture}[scale=0.8]
\begin{axis}[
xlabel=$P\; atma$,
ylabel=$\mu_w\; cP$,
legend pos=north west,
title = McCain]
\addplot table [y=T_1_20, x=P]{\MuwDataFile};
\addlegendentry{$T = 20$ С}
\addplot table [y=T_1_60, x=P]{\MuwDataFile};
\addlegendentry{$T = 60$ С}
\addplot table [y=T_1_100, x=P]{\MuwDataFile};
\addlegendentry{$T = 100$ С}
\addplot table [y=T_1_140, x=P]{\MuwDataFile};
\addlegendentry{$T = 140$ С}
\end{axis}
\end{tikzpicture}

\subsection{PVT\_Rhoo\_kgm3 – плотность нефти}
Функция вычисляет значение плотности нефти при заданных термобарических условиях. Результат расчета имеет размерность кг/м3. 


\putlisting{listings/PVT_Rhoo_kgm3.txt}

\newcommand{\RhooDataFile}{data/Rhoo_P_data.txt}
\begin{tikzpicture}[scale=0.8]
\begin{axis}[
xlabel=$P\; atma$,
ylabel=$\rho_o\; kg/m^3$,
legend pos=north east,
title=Standing]
\addplot table [y=T_0_20, x=P]{\RhooDataFile};
\addlegendentry{$T = 20$ С}
\addplot table [y=T_0_60, x=P]{\RhooDataFile};
\addlegendentry{$T = 60$ С}
\addplot table [y=T_0_100, x=P]{\RhooDataFile};
\addlegendentry{$T = 100$ С}
\addplot table [y=T_0_140, x=P]{\RhooDataFile};
\addlegendentry{$T = 140$ С}
\end{axis}
\end{tikzpicture}
\begin{tikzpicture}[scale=0.8]
\begin{axis}[
xlabel=$P\; atma$,
ylabel=$\rho_o \; kg/m^3$,
legend pos=north east,
title = McCain]
\addplot table [y=T_1_20, x=P]{\RhooDataFile};
\addlegendentry{$T = 20$ С}
\addplot table [y=T_1_60, x=P]{\RhooDataFile};
\addlegendentry{$T = 60$ С}
\addplot table [y=T_1_100, x=P]{\RhooDataFile};
\addlegendentry{$T = 100$ С}
\addplot table [y=T_1_140, x=P]{\RhooDataFile};
\addlegendentry{$T = 140$ С}
\end{axis}
\end{tikzpicture}



\subsection{PVT\_Rhog\_kgm3 – плотность газа}
\putlisting{listings/PVT_Rhog_kgm3.txt}
	
\newcommand{\RhogDataFile}{data/Rhog_P_data.txt}
\begin{tikzpicture}[scale=0.8]
\begin{axis}[
xlabel=$P\; atma$,
ylabel=$\rho_g\; kg/m^3$,
legend pos=north west,
title=Standing]
\addplot table [y=T_0_20, x=P]{\RhogDataFile};
\addlegendentry{$T = 20$ С}
\addplot table [y=T_0_60, x=P]{\RhogDataFile};
\addlegendentry{$T = 60$ С}
\addplot table [y=T_0_100, x=P]{\RhogDataFile};
\addlegendentry{$T = 100$ С}
\addplot table [y=T_0_140, x=P]{\RhogDataFile};
\addlegendentry{$T = 140$ С}
\end{axis}
\end{tikzpicture}
\begin{tikzpicture}[scale=0.8]
\begin{axis}[
xlabel=$P\; atma$,
ylabel=$\rho_g \; kg/m^3$,
legend pos=north west,
title = McCain]
\addplot table [y=T_1_20, x=P]{\RhogDataFile};
\addlegendentry{$T = 20$ С}
\addplot table [y=T_1_60, x=P]{\RhogDataFile};
\addlegendentry{$T = 60$ С}
\addplot table [y=T_1_100, x=P]{\RhogDataFile};
\addlegendentry{$T = 100$ С}
\addplot table [y=T_1_140, x=P]{\RhogDataFile};
\addlegendentry{$T = 140$ С}
\end{axis}
\end{tikzpicture}

\subsection{PVT\_Rhow\_kgm3 – плотность воды}
\putlisting{listings/PVT_Rhow_kgm3.txt}

\newcommand{\RhowDataFile}{data/Rhow_P_data.txt}
\begin{tikzpicture}[scale=0.8]
\begin{axis}[
xlabel=$P\; atma$,
ylabel=$\rho_w\; kg/m^3$,
legend pos=north east,
title=Standing]
\addplot table [y=T_0_20, x=P]{\RhowDataFile};
\addlegendentry{$T = 20$ С}
\addplot table [y=T_0_60, x=P]{\RhowDataFile};
\addlegendentry{$T = 60$ С}
\addplot table [y=T_0_100, x=P]{\RhowDataFile};
\addlegendentry{$T = 100$ С}
\addplot table [y=T_0_140, x=P]{\RhowDataFile};
\addlegendentry{$T = 140$ С}
\end{axis}
\end{tikzpicture}
\begin{tikzpicture}[scale=0.8]
\begin{axis}[
xlabel=$P\; atma$,
ylabel=$\rho_w \; kg/m^3$,
legend pos=north east,
title = McCain]
\addplot table [y=T_1_20, x=P]{\RhowDataFile};
\addlegendentry{$T = 20$ С}
\addplot table [y=T_1_60, x=P]{\RhowDataFile};
\addlegendentry{$T = 60$ С}
\addplot table [y=T_1_100, x=P]{\RhowDataFile};
\addlegendentry{$T = 100$ С}
\addplot table [y=T_1_140, x=P]{\RhowDataFile};
\addlegendentry{$T = 140$ С}
\end{axis}
\end{tikzpicture}

\subsection{PVT\_Z – коэффициент сверхсжимаемости газа}

Функция позволяет рассчитать коэффициент сверхсжимаемости газа. 


$$ PV = z \nu RT  $$



\putlisting{listings/PVT_Z.txt}

\newcommand{\zDataFile}{data/Z_P_data.txt}
\begin{tikzpicture}[scale=0.8]
\begin{axis}[
xlabel=$P\; atma$,
ylabel=$Z$,
legend pos=south west,
title=Standing]
\addplot table [y=T_0_20, x=P]{\zDataFile};
\addlegendentry{$T = 20$ С}
\addplot table [y=T_0_60, x=P]{\zDataFile};
\addlegendentry{$T = 60$ С}
\addplot table [y=T_0_100, x=P]{\zDataFile};
\addlegendentry{$T = 100$ С}
\addplot table [y=T_0_140, x=P]{\zDataFile};
\addlegendentry{$T = 140$ С}
\end{axis}
\end{tikzpicture}
\begin{tikzpicture}[scale=0.8]
\begin{axis}[
xlabel=$P\; atma$,
ylabel=$Z$,
legend pos=south west,
title = McCain]
\addplot table [y=T_1_20, x=P]{\zDataFile};
\addlegendentry{$T = 20$ С}
\addplot table [y=T_1_60, x=P]{\zDataFile};
\addlegendentry{$T = 60$ С}
\addplot table [y=T_1_100, x=P]{\zDataFile};
\addlegendentry{$T = 100$ С}
\addplot table [y=T_1_140, x=P]{\zDataFile};
\addlegendentry{$T = 140$ С}
\end{axis}
\end{tikzpicture}

\subsection{PVT\_Qmix\_m3day – расход газожидкостной смеси}

Функция позволяет рассчитать объемный расход газожидкостной смеси при заданных термобарических условиях. 

$$Q_{mix} = Q_w B_w(P,T) + Q_o B_o(P,T)  + Q_o  (R_p - R_s(P,T)) B_g(P,T) $$



\putlisting{listings/PVT_Qmix_m3day.txt}

\section{Сепарация газа в скважине}

\subsection{MF\_SeparNat\_d – естественная сепарация газа}

Функция рассчитывает естественную сепарацию газа в стволе скважине с использованием корреляции Маркеса \cite{Marquez_2003} . Результат - безразмерная величина в диапазоне от 0 до 1. 

\putlisting{listings/MF_SeparNat_d.txt}

\subsection{MF\_SeparTotal\_d – естественная сепарация газа}

Функция рассчитывает полную сепарацию газа в стволе скважине по известным значениям естественной сепарации газа и коэффициента сепарации газасепаратора. Результат - безразмерная величина в диапазоне от 0 до 1. 

\putlisting{listings/MF_SeparTotal_d.txt}

\subsection{MF\_GasFraction\_d – доля газа в потоке}
Функция расчёта доли свободного газа в потоке (без учёта проскальзывания) в зависимости от термобарических условий для заданного флюида. 
В отличии от функций PVT учитывается обводнённость.
\putlisting{listings/MF_GasFraction_d.txt}

\subsection{MF\_PGasFraction\_atma – целевое давления для заданной доли газа в потоке}
Функция расчёта давления при котором достигается заданная доля свободного газа в потоке (без учета проскальзывания) . 
В отличии от функций PVT учитывается обводнённость.
Следует учитывать, что при вызове функции пересчитывается состояние смеси с различными термобарическими условиями.
\putlisting{listings/MF_PGasFraction_atma.txt}

\subsection{MF\_RpGasFraction\_m3m3 – целевой газовый фактор для заданной доли газа в потоке}
Функция расчёта давления при котором достигается заданная доля свободного газа в потоке (без учета проскальзования) . 
В отличии от функций PVT учитывается обводненность.
Следует учитывать, что при вызове функции пересчитывается состояние смеси с различными термобарическими условиями.
\putlisting{listings/MF_RpGasFraction_m3m3.txt}

\section{Расчёт многофазного потока в штуцере}

\subsection{Модель потока через штуцер}

Тут надо будет нарисовать схему штуцера и пояснить что и как называется в коде. 

Что такое Ап и Даун

У элемента гидравлического потока есть три ключевых параметра 


\subsection{MF\_dPChokeUp\_atm – Расчет перепада давления в штуцере против потока}
Функция позволяет рассчитать по известному линейному давлению и дебиту буферное давление. 
Функция возвращает давление и температуру в виде массива.
\begin{listing}[H]
	\begin{minted}{vb.net}
Public Function MF_dPChokeUp_atm(dPipe_mm, dchoke_mm, Pdown_atm,
		    Qliq_m3day, WCT_perc, _
			Optional Tchoke_C = 20, _
			Optional gg = const_gg_default, _
			Optional go = const_go_default, _
			Optional gw = const_gw_default, _
			Optional Rsb_m3m3 = const_Rsb_default, _
			Optional Rp_m3m3 As Double = -1, _
			Optional Pb_atm As Double = -1, _
			Optional Tres_C As Double = const_Tres_default, _
			Optional Bob_m3m3 As Double = -1, _
			Optional Muob_cP As Double = -1, _
			Optional PVTcorr = StandingBased, _
			Optional Ksep_fr As Double = 0, _
			Optional PKsep_atm As Double = -1, _
			Optional TKsep_C As Double = -1)

' dPipe_mm      - диаметр трубы до и после штуцера
' dChoke_mm     - диаметр штуцера (эффективный)
' Pdown_atm     - давление на выходе (низкой стороне)
' Qliq_m3day    - дебит жидкости в пов условиях
' WCT_perc      - обводненность
' данные PVT
	
	\end{minted}
	\caption{Объявление функции перепада давления в штуцере}
	\label{lst:codedPChokeUp}
\end{listing}



\newpage
\subsection{MF\_QChoke\_m3day – функция расчёта дебита жидкости через штуцер}
Функция позволяет рассчитать по известному буферному давлению и линейному давлению дебит.
Функция возвращает давление и температуру в виде массива.

\putlisting{listings/MF_dPchoke_atm.txt}

\newpage
\section{Расчет многофазного потока в трубе}



\subsection{MF\_dPpipe\_atma – расчёт перепада давления в трубе}

Функция позволяет рассчитать перепад давления в участке трубопровода. 

Функция возвращает давление и температуру в виде массива.

\putlisting{listings/MF_dPpipe_atma.txt}

Ниже на рисунке приведены результаты расчёта кривой оттока (перепада давления в вертикальной трубе) для различных корреляций реализованных в \unf.

\newcommand{\dPipeDataFile}{data/dPipe.txt}
\begin{tikzpicture}[scale=1]
\begin{axis}[
width=14cm,
height=10cm,
xlabel=$Q\; m^3 / day$,
ylabel=$P_{wf} \; atma$,
legend pos=south east,
title=Pipe Pressure Drop]
\addplot table [y=P_0, x=Q]{\dPipeDataFile};
\addlegendentry{Beggs Brill}
\addplot table [y=P_1, x=Q]{\dPipeDataFile};
\addlegendentry{Ansari}
\addplot table [y=P_2, x=Q]{\dPipeDataFile};
\addlegendentry{Unified}
\addplot table [y=P_3, x=Q]{\dPipeDataFile};
\addlegendentry{Gray}
\addplot table [y=P_4, x=Q]{\dPipeDataFile};
\addlegendentry{Hagedorn Brown}
\addplot table [y=P_5, x=Q]{\dPipeDataFile};
\addlegendentry{Sakharov Mokhov}
\end{axis}
\end{tikzpicture}



\newpage
\section{Расчет многофазного потока в пласте}
\newpage
\subsection{IPR\_PI\_sm3dayatm – расчёт продуктивности}
Функция позволяет рассчитать коэффициент продуктивности скважины.
\begin{listing}[H]
	\begin{minted}{vb.net}
' расчёт продуктивности
Public Function IPR_PI_sm3dayatm(Qtest_m3day, Pwftest_atm, Pr_atm, _
Optional WCT_perc As Double = 0, Optional Pb_atm As Double = -1)
	'
	' Qtest_m3day   - тестовый дебит скважины
	' Pwftest_atm   - тестовое забойное давление
	' Pr_atm        - пластовое давление, атм
	'
	' необязательные параметры
	' WCT_perc      - обводненность
	' Pb_atm        - давление насыщения
	'
	\end{minted}
	\caption{Объявление функции расчёта продуктивности}
	\label{lst:codedIPR_PI}
\end{listing}
\newpage
\subsection{IPR\_Pwf\_atm – расчёт дебита по давлению и продуктивности}
Функция позволяет рассчитать дебит жидкости скважины по известным значениям давления и продуктивности.

\begin{listing}[H]
	\begin{minted}{vb.net}
' расчёт дебита по давлению и продуктивности
Public Function IPR_Pwf_atm(PI_m3dayatm, Pr_atm, Ql_m3day, _
Optional WCT_perc As Double = 0, Optional Pb_atm As Double = -1)
'
' PI_m3dayatm   - коэффициент продуктивности
' Pr_atm        - пластовое давление, атм
' Ql_m3day      - дебит жидкости скважины на поверхности
'
' необязательные параметры
' WCT_perc      - обводненность
' Pb_atm        - давление насыщения
'
	\end{minted}
	\caption{Объявление функции расчёта дебита по давлению и продуктивности}
	\label{lst:codedIPR_Pwf}
\end{listing}
\newpage
\subsection{IPR\_Ql\_sm3Day – расчёт дебита по давлению и продуктивности}
Функция позволяет рассчитать дебита по давлению и продуктивности.
\begin{listing}[H]
	\begin{minted}{vb.net}
' расчёт дебита по давлению и продуктивности
Public Function IPR_Ql_sm3Day(PI_m3dayatm, Pr_atm, Pwf_atm, _
Optional WCT_perc As Double = 0, Optional Pb_atm As Double = -1)
'
' PI_m3dayatm   - коэффициент продуктивности
' Pr_atm        - пластовое давление, атм
' Pwf_atm       - забойное давление
'
' необязательные параметры
' WCT_perc      - обводненность
' Pb_atm        - давление насыщения
'

	\end{minted}
	\caption{Объявление функции расчёта дебита по давлению и продуктивности}
	\label{lst:codedIPR_Ql}
\end{listing}


\newpage
\chapter{Функции модуля  «u7\_Excel\_functions\_ESP»}
В этом модули приведены интерфейсные функции Excel (функции, которые можно вызывать непосредственно с листа Excel) для расчёта параметров работы УЭЦН - установки электрического центробежного насоса. 

УЭЦН состоит из следующих основных конструктивных элементов:
\begin{itemize}
	\item ЦН - центробежный насос. Модуль обеспечивающий перекачку жидкости.
	\item ПЭД - погружной электрический двигатель. Модуль обеспечивающий преобразование электрической энергии, поступающий к УЭЦН по кабелю в механическую энергию вращения вала.
	\item ГС - газосепаратор или приемный модуль. Модуль обеспечивающий забор пластовой жидкости из скважины и подачу ее в насос. При этом центробежный газосепаратор способе отделить часть свободного газа в потоке и направить его в межтрубное пространство скважины.
	\item вал - узел передающий энергию от погружного электрического двигателя (ПЭД) к остальным узлам установки, в том числе к центробежному насосу.
\end{itemize}

Задача расчета УЭЦН обычно сводится к следующим:
\begin{itemize}
	\item Прямая задача - по заданным значения дебита жидкости скважины,  давлению на приеме, напряжению питания УЭЦН на поверхности найти давление на выкиде насоса, потребляему электрическую мощность, потребляемый ток установки, КПД всей системы и отдельных узлов системы
	\item Обратная задача - по данным контроля параметров работы УЭЦН на поверхности - потребляемому току, напряжению питания частоте подаваемого напряжения, данным по конструкции УЭЦН и скважины найти дебит жидкости и обводнённость по скважине, давление на приеме и забойное давление.
	\item Задача узлового анализа - по данным конструкции скважины, параметров работы погружного оборудования оценить дебит по жидкости скважины при заданным параметрах работы УЭЦН или при из изменении. К этому типу задач относится задача подбора погружного оборудования для достижения заданных условий эксплуатации 
	
\end{itemize}

Для расчёта УЭЦН требуется рассчитать гидравлические параметры работы ЦН и электромеханические параметры ПЭД

\section{Гидравлический расчет центробежного насоса (ЦН)}

Расчет выполняется на основе паспортных характеристик ЦН. 

\section{Электромеханический расчёт погружного электрического двигателя ПЭД}
Рассматривается асинхронный электрический двигатель. 

Погружные асинхронные электрические двигатели для добычи нефти выполяются трехфазными. 

Впервые конструкция трёхфазного асинхронного двигателя была разработана, создана и опробована русским инженером М. О. Доливо-Добровольским в 1889-91 годах. Демонстрация первых двигателей состоялась на Международной электротехнической выставке во Франкфурте на Майне в сентябре 1891 года. На выставке было представлено три трёхфазных двигателя разной мощности. Самый мощный из них имел мощность 1.5 кВт и использовался для приведения во вращение генератора постоянного тока. Конструкция асинхронного двигателя, предложенная Доливо-Добровольским, оказалась очень удачной и является основным видом конструкции этих двигателей до настоящего времени.

За прошедшие годы асинхронные двигатели нашли очень широкое применение в различных отраслях промышленности и сельского хозяйства. Их используют в электроприводе металлорежущих станков, подъёмно-транспортных машин, транспортёров, насосов, вентиляторов. Маломощные двигатели используются в устройствах автоматики.

Широкое применение асинхронных двигателей объясняется их достоинствами по сравнению с другими двигателями: высокая надёжность, возможность работы непосредственно от сети переменного тока, простота обслуживания.

Для расчёта электромеханических параметров погружных электрических двигателей полезно понимать теоретические основы их работы. Теория работы погружных асинхронных двигателей не отличаем от теории применимой к двигателям применяемым на поверхности. Далее кратко изложены основные положения теории. 

Трехфазная цепь является частным случаем многофазных систем электрических цепей, представляющих собой совокупность электрических цепей, в которых действуют синусоидальные ЭДС одинаковой частоты, отличающиеся по фазе одна от другой и создаваемые общим источником энергии.
Переменный ток протекающий по трехфазной цели характеризуется следующими параметрами:

\begin{itemize}
	\item Фазное напряжение $U_A, U_B, U_C $ - напряжение между линейным проводом и нейтралью
	\item Линейное напряжение $U_{AB}, U_{BC}, U_{CA} $ - напряжение между одноименными выводами разных фаз
	\item Фазный ток $I_{phase}$ – ток в фазах двигателя.
	\item Линейный ток $I_{line}$ – ток в линейных проводах.
	\item $ \cos \varphi $ - коэффициент мощности, где $ \varphi$ величина сдвига по фазе между напряжением и током 
\end{itemize}

Подключение двигателя к цепи трехфазного тока может быть выполнено по схеме "звезда" или "треугольник".

Тут нужен рисунок  

Для схемы звезда фазное напряжение меньше линейного в $\sqrt{3}$ раз.

$$ U_{AB} = \sqrt{3} U_{A} $$
$$ I_{phase} = I_{line} $$

Для схемы треугольник 

$$ U_{AB} =  U_{A} $$
$$ I_{line} =\sqrt{3} I_{phase} $$


В погружных двигателях обычно применяет схема подключения звезда. Эта схема обеспечивает более низкое напряжение в линии, что способствует повышению КПД передачи энергии по длинному кабелю. Еще есть причины?
При схеме подключения звезда токи в линии и в фазной обмотке статора двигателя совпадают, поэтому значение тока обозначают $I$ не указывая индекс в явном виде. Поскольку линейное напряжения проще измерить и легче контролировать параметры трехфазного двигателя обычно заданию линейный. в частности номинальное напряжение питания двигателя это линейное напряжение (напряжение между фазами). Далее линейное напряжение будет обозначать без индекса как $U$

Активная электрическая мощность в трехфазной цепи задается выражением 
$$ P= \sqrt{3}U I \cos \varphi$$

Реактивная мощность 
$$ Q= \sqrt{3}U I \sin \varphi$$

Соответственно полная мощность 
$$ S= \sqrt{3}U I $$

\subsection{ Устройство трёхфазной асинхронной машины}
Неподвижная часть машины называется статор, подвижная – ротор. Обмотка статора состоит из трёх отдельных частей, называемых фазами.

При подаче переменного напряжения и тока на обмотки статора внутри статора формируется вращающееся магнитное поле. Частота вращения магнитного поля совпадает с частотой питающего напряжения. 

Магнитный поток $\Phi $ и напряжение подаваемое на статор связаны приближенном соотношением 
$$ U_1 \approx E_1 = 4.44 w_1 k_1 f \Phi $$
где 

 $\Phi$ -  магнитный поток;
 
 $U_1$ -	напряжение в одной фазе статора;
 
 $f$   - частота сети;
 
 $E_1$	- ЭЦН в фазе статора;
 
 $w_1$ - число витков одной фазы обмотки статора;
 
 $k_1$  - обмоточный коэффициент.
   
Из этого выражения следует, что магнитный поток $\Phi $ в асинхронной машине не зависит от её режима работы, а при заданной частоте сети $f$ зависит только от действующего значения приложенного напряжения $U_1$


Для ЭДС ротора можно записать выражение 

$$  E_2 = 4.44 w_2 k_2 f S \Phi $$

где 


$S$ - величина скольжения (проскальзования);

$E_2$	- ЭЦН в фазе ротора;

$w_2$ - число витков одной фазы обмотки ротора;

$k_2$ - обмоточный коэффициент ротора.

ЭДС, наводимая в обмотке ротора, изменяется пропорционально скольжению и в режиме двигателя имеет наибольшее значение в момент пуска в ход.
Для тока ротора в общем случае можно получить такое соотношение

$$  I_2 = \frac{E_2 S}{\sqrt{R_2^2+(S X_2^2)}} $$

где 

$R_2$ -  активное сопротивление обмотки ротора, связанное с потерями на нагрев обмотки;  

$X_2 = 2 \pi f L_2$ - индуктивное сопротивление обмотки неподвижного ротора, связанное с потоком рассеяния;

Отсюда следует, что ток ротора зависит от скольжения и возрастает при его увеличении, но медленнее, чем ЭДС.

Для асинхронного двигателя можно получить следующее выражение для механического момента 

$$ M = \frac{1}{4.44 w_2 k_2 k_T^2 f} \frac{U_1^2 R_2 S}{R_2^2 + (S X_2^2)^2}$$

где 

$k_T = \frac{E_1}{E_2} = \frac{w_1 k_1}{w_2 k_2}$ - коэффициент трансформации асинхронной машины

Из полученного выражения для электромагнитного момента следует, что он сильно зависит от подведённого напряжения ($M \sim U_1^2$). При снижении, например, напряжения на 10\%, электромагнитный момент снизится на 19\% $M \sim (0,9U_1)^2=0.81 U_1^2)$. Это является одним из недостатков асинхронных двигателей. 




\newpage
\chapter{Функции модуля  «tr\_mdlTecRegimes»}
Одна из первых реализаций расчётных модулей \unf была создана для проведения расчётов потенциала добычи нефти в форме технологического режима добывающих скважин. Расчёты были реализованы в начале 2000х годов. Расчётная форма оказалась удобной для практического применения и со временем алгоритмы расчёта распространились по разным компаниям и широко использовались.
\section{Технологический режим добывающих скважин}

 

Для обеспечения обратной совместимости расчётов в \unf заложены основные функции расчёта из технологического режима работы скважин. У функций изменены названия функций и имена аргументов, однако алгоритмы расчётов оставлены без изменений.

\newpage
\subsection{tr\_Pwf\_calc\_atma – расчёт забойного давления по динамическому уровню}

Функция рассчитывает забойное давление добывающей нефтяной скважины. Расчёт выполняется по известному значению затрубного давления и динамическому уровню. \cite{Khasanov_TR_2006}

Результат расчёта - абсолютное значение забойного давления. 

Расчёт выполняется по модифицированной корреляции Хасана-Кабира оптимизированной для скорости вычисления как для интервала выше насоса в межтрубном пространстве, так и для участка ниже насоса. При расчёте пренебрегается трением в потоке и используются упрощённые PVT зависимости, что позволило получить результат в аналитическом виде и ускорить расчёты. [ссылку надо будет привести когда то] 

Функция позволяет учесть удлинения скважин для забоя, глубины спуска насоса, и динамического уровня. Два последних значения являются опциональными и могут быть опущены при проведении расчёта. 

\putlisting{listings/tr_Pwf.txt}



\subsection{tr\_Pwf\_calc\_Pin\_atma – расчёт забойного давления по давлению на приеме}
Функция рассчитывает забойное давление добывающей нефтяной скважины по известному значению давления на приёме насоса. 

Результат расчёта - абсолютное значение забойного давления. 

Расчёт выполняется по модифицированной корреляции Хасана-Кабира оптимизированной для скорости вычисления для участка ниже насоса. При расчёте пренебрегается трением в потоке и используются упрощённые PVT зависимости, что позволило получить результат в аналитическом виде и ускорить расчёты. [ссылку надо будет привести когда то] 

Функция позволяет учесть удлинения скважин для забоя, глубины спуска насоса. Последнее значение являются опциональными и могут быть опущены при проведении расчёта. 
\putlisting{listings/tr_Pwf_calc_Pin_atma.txt}


\subsection{tr\_Ppump\_calc\_atma – расчёт давления на приеме по динамическому уровню}
Функция рассчитывает давление на приёме насоса добывающей нефтяной скважины по известному значению затрубного давления и динамическому уровню. 

Расчёт выполняется по модифицированной корреляции Хасана-Кабира оптимизированной для скорости вычисления для участка выше насоса. При расчёте пренебрегается трением в потоке и используются упрощённые PVT зависимости, что позволило получить результат в аналитическом виде и ускорить расчёты. [ссылку надо будет привести когда то]. Значение коэффициента сепарации используется для оценки объёмного расхода газа в межтрубном пространстве. 

Результат расчёта - абсолютное значение давления на приёме насоса. 
\putlisting{listings/tr_Ppump_calc_atma.txt}



\subsection{tr\_Potential\_Pwf\_atma – расчёт целевого забойного давления по доле газа}
Функция рассчитывает целевое забойное давление добывающей нефтяной скважины при котором достигается заданная доля газа в потоке.

Результат расчёта - абсолютное значение забойного давления. 
\putlisting{listings/tr_Potential_Pwf_atma.txt}

\subsection{tr\_BB\_Pwf\_atma – расчёт забойного давления фонтанирующей скважины по буферному давлению}
Функция рассчитывает забойное давление фонтанирующей добывающей скважины по известному значению буферного давления. Расчет выполняется по корреляции Бегсса Брилла. 

Расчет отличается рядом упрощений - из PVT свойств используется только значение газового фактора - давление насыщения и объемный коэффициент газа вычисляются по корреляциям. 

В отличии от расчёта скважин с насосом в корреляции Беггса Брилла учитывается наличие трения. Хотя для низких дебитов эта корреляция может давать завышенные значения перепада давления. 

Для расчётов рекомендуется использовать функцию \unf реализующую аналогичную функциональность с меньшим набором допущений

Результат расчёта - абсолютное значение забойного давления. 
\putlisting{listings/tr_BB_Pwf_atma.vb}

\subsection{tr\_BB\_Pwf\_Pin\_atma – расчёт забойного давления по давлению на приеме по корреляции Беггса-Брилла}
Функция рассчитывает забойное давление  добывающей скважины по известному значению давления на приёме. Расчёт выполняется по корреляции Бегсса-Брилла. 
Расчёт отличается рядом упрощений - из PVT свойств используется только значение газового фактора - давление насыщения и объёмный коэффициент газа вычисляются по корреляциям. 

В отличии от расчёта скважин с насосом в корреляции Беггса Брилла учитывается наличие трения. Хотя для низких дебитов эта корреляция может давать завышенные значения перепада давления. 

Для расчётов рекомендуется использовать функцию \unf реализующую аналогичную функциональность с меньшим набором допущений

Результат расчёта - абсолютное значение забойного давления. 

\putlisting{listings/tr_BB_Pwf_Pin_atma.vb}

\cite{HasanKabir_HeatTransfer_2002}

\bibliography{biblio}{}
\bibliographystyle{plain}


1.	Brill, J.P., Mukherjee H. Multiphase flow in wells. SPE monograph.
2.	Vogel, J.V.: “Inflow Peformance Relationships for Solution-Gas Drive Wells,” JPT (Jan. 1968) 83-92; Trans., AIME, 243.
3.	Standing, M.B.: “Concerning the Calculation of Inflow Performance of Wells Producing from Solution Gas Reservoirs,” JPT (Sept. 1971) 1141-42
4.	Standing, M.B.: “Inflow Performance Relationships for Damaged Wells Producing by Solution Gas Reservoirs,” JPT (Sept. 1971) 1141-42
5.	Dias-Couto, L.E. and Golan, M.: “General Inflow Performance Relationship for for Solution-Gas Reservoir Wells,” JPT (Feb. 1982) 285-88
6.	R.G. Camacho-V., R. Raghavan: “Inflow performance Relationships for Solution-Gas-Drive Reservoirs,”  JPT (May 1989) 541-50
7.	Fetkovich, M.G.: “The Isochronal Testing of Oil Wells,” paper SPE 4529 presented at the 1973 SPE Annual Technical  Conference and Exhibition, Las Vegas, Sept. 30-Oct. 3.
8.	Hasan, A.R., Kabir C.S.: “A Study of Multiphase Flow Behavior in Vertical Wells,” SPE PE (May 1988) 263-72
9.	Hasan, A.R., Kabir C.S.: “A Study of Multiphase Flow Behavior in Vertical Wells: Part II – Field Application,” Paper SPE 15139, 1986
10.	Hasan, A.R., Kabir C.S.: “Two-phase Flow Correlations as Applied to Pumping well Testing,” Paper SPE 21727, 1991
11.	 Guerra, T., Yildiz T.: “A Simple Approximate Method to Predict Inflow Performance of Selectively Perforated Vertical Wells,” paper SPE 89414.




\end{document}
