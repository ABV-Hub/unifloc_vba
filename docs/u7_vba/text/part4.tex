\chapter{Функции модуля  «tr\_mdlTecRegimes»}
Одна из первых реализаций расчётных модулей \unf была создана для проведения расчётов потенциала добычи нефти в форме технологического режима добывающих скважин. Расчёты были реализованы в начале 2000х годов. Расчётная форма оказалась удобной для практического применения и со временем алгоритмы расчёта распространились по разным компаниям и широко использовались.
\section{Технологический режим добывающих скважин}

 

Для обеспечения обратной совместимости расчётов в \unf заложены основные функции расчёта из технологического режима работы скважин. У функций изменены названия функций и имена аргументов, однако алгоритмы расчётов оставлены без изменений.

\newpage
\subsection{tr\_Pwf\_calc\_atma – расчёт забойного давления по динамическому уровню}

Функция рассчитывает забойное давление добывающей нефтяной скважины. Расчёт выполняется по известному значению затрубного давления и динамическому уровню. \cite{Khasanov_TR_2006}

Результат расчёта - абсолютное значение забойного давления. 

Расчёт выполняется по модифицированной корреляции Хасана-Кабира оптимизированной для скорости вычисления как для интервала выше насоса в межтрубном пространстве, так и для участка ниже насоса. При расчёте пренебрегается трением в потоке и используются упрощённые PVT зависимости, что позволило получить результат в аналитическом виде и ускорить расчёты. [ссылку надо будет привести когда то] 

Функция позволяет учесть удлинения скважин для забоя, глубины спуска насоса, и динамического уровня. Два последних значения являются опциональными и могут быть опущены при проведении расчёта. 

%\putlisting{listings/tr_Pwf.txt}



\subsection{tr\_Pwf\_calc\_Pin\_atma – расчёт забойного давления по давлению на приеме}
Функция рассчитывает забойное давление добывающей нефтяной скважины по известному значению давления на приёме насоса. 

Результат расчёта - абсолютное значение забойного давления. 

Расчёт выполняется по модифицированной корреляции Хасана-Кабира оптимизированной для скорости вычисления для участка ниже насоса. При расчёте пренебрегается трением в потоке и используются упрощённые PVT зависимости, что позволило получить результат в аналитическом виде и ускорить расчёты. [ссылку надо будет привести когда то] 

Функция позволяет учесть удлинения скважин для забоя, глубины спуска насоса. Последнее значение являются опциональными и могут быть опущены при проведении расчёта. 
%\putlisting{listings/tr_Pwf_calc_Pin_atma.txt}


\subsection{tr\_Ppump\_calc\_atma – расчёт давления на приеме по динамическому уровню}
Функция рассчитывает давление на приёме насоса добывающей нефтяной скважины по известному значению затрубного давления и динамическому уровню. 

Расчёт выполняется по модифицированной корреляции Хасана-Кабира оптимизированной для скорости вычисления для участка выше насоса. При расчёте пренебрегается трением в потоке и используются упрощённые PVT зависимости, что позволило получить результат в аналитическом виде и ускорить расчёты. [ссылку надо будет привести когда то]. Значение коэффициента сепарации используется для оценки объёмного расхода газа в межтрубном пространстве. 

Результат расчёта - абсолютное значение давления на приёме насоса. 
%\putlisting{listings/tr_Ppump_calc_atma.txt}



\subsection{tr\_Potential\_Pwf\_atma – расчёт целевого забойного давления по доле газа}
Функция рассчитывает целевое забойное давление добывающей нефтяной скважины при котором достигается заданная доля газа в потоке.

Результат расчёта - абсолютное значение забойного давления. 
%\putlisting{listings/tr_Potential_Pwf_atma.txt}

\subsection{tr\_BB\_Pwf\_atma – расчёт забойного давления фонтанирующей скважины по буферному давлению}
Функция рассчитывает забойное давление фонтанирующей добывающей скважины по известному значению буферного давления. Расчет выполняется по корреляции Бегсса Брилла. 

Расчет отличается рядом упрощений - из PVT свойств используется только значение газового фактора - давление насыщения и объемный коэффициент газа вычисляются по корреляциям. 

В отличии от расчёта скважин с насосом в корреляции Беггса Брилла учитывается наличие трения. Хотя для низких дебитов эта корреляция может давать завышенные значения перепада давления. 

Для расчётов рекомендуется использовать функцию \unf реализующую аналогичную функциональность с меньшим набором допущений

Результат расчёта - абсолютное значение забойного давления. 
%\putlisting{listings/tr_BB_Pwf_atma.vb}

\subsection{tr\_BB\_Pwf\_Pin\_atma – расчёт забойного давления по давлению на приеме по корреляции Беггса-Брилла}
Функция рассчитывает забойное давление  добывающей скважины по известному значению давления на приёме. Расчёт выполняется по корреляции Бегсса-Брилла. 
Расчёт отличается рядом упрощений - из PVT свойств используется только значение газового фактора - давление насыщения и объёмный коэффициент газа вычисляются по корреляциям. 

В отличии от расчёта скважин с насосом в корреляции Беггса Брилла учитывается наличие трения. Хотя для низких дебитов эта корреляция может давать завышенные значения перепада давления. 

Для расчётов рекомендуется использовать функцию \unf реализующую аналогичную функциональность с меньшим набором допущений

Результат расчёта - абсолютное значение забойного давления. 

%\putlisting{listings/tr_BB_Pwf_Pin_atma.vb}

