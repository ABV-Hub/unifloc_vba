\chapter{Макросы VBA для проведения расчётов}

Расчёты с использованием \unf выполняются с использованием макросов написанных на языке программирования Visual Basic for Application (VBA) в среде Excel.

Для использования макросов не требуется программировать, достаточно уметь вызывать необходимые макросы из Excel. Тем не менее макросы \unf могут быть использованы для написания собственных подпрограмм или модифицированы для достижения необходимых целей. Владение навыками программирования и изучения исходного кода макросов может оказаться чрезвычайно полезным. 

Исходный код расчётных модулей находится в отдельном файле - надстройке Excel файле с расширением.xlam. Для использования макросов данная надстройка должна быть установлена на компьютере, на котором проводятся расчёты. Подробное описание процедуры установки надстройки можно найти на сайте microsoft по ключевым словам \href{https://support.office.com/ru-ru/article/%D0%94%D0%BE%D0%B1%D0%B0%D0%B2%D0%BB%D0%B5%D0%BD%D0%B8%D0%B5-%D0%B8-%D1%83%D0%B4%D0%B0%D0%BB%D0%B5%D0%BD%D0%B8%D0%B5-%D0%BD%D0%B0%D0%B4%D1%81%D1%82%D1%80%D0%BE%D0%B5%D0%BA-%D0%B2-excel-0af570c4-5cf3-4fa9-9b88-403625a0b460}{"добавление и удаление надстроек в Excel"}.

Для активации надстройки 
\begin{enumerate}
	\item На вкладке Файл выберите команду Параметры, а затем — категорию Надстройки.
	\item В поле Управление выберите пункт Надстройки Excel, а затем нажмите кнопку Перейти. Откроется диалоговое окно Надстройки.
	\item Чтобы установить и активировать надстройку Унифлок 7.1, нажмите кнопку Обзор (в диалоговом окне Надстройки), найдите надстройку, а затем нажмите кнопку ОК.
	\item Надстройка появится в списке надстроек. Галочка активации надстройки должна быть установлена
\end{enumerate}	
После установки и активации надстройки, встроенными в нее макросами можно будет пользоваться в любой книге Excel на данным компьюетере. При переносе расчётный файлов на другой комппьютер для сохранения их работоспособности должна быть передана и установлена и надстройка. 


В некоторых случаях может быть удобен альтернативный способ работы с надстройкой, не требующий ее установки на компьютере. Это бывает удобно, когда версия настройки часто меняется. Для этого необходимо открыть файл надстройки непосредственно в Excel, например двойным щелчком по файлу надстройки в проводнике. При этом Excel откроется, но никаких документов в нем не появится. Но сама надстройка будет загружена и готова к использованию любым файлом открытом в этой копии Excel. Следует обратить внимание, что при таком варианте работы с надстройкой при переносе сохраненных файлом между компьютерами при открытии файла может возникать запрос, что связанный файл надстройки не найден на новом компьютере. В этом случае в окне запроса следует выбрать кнопку изменить и указать правильное положение файла надстройки.


\section{Запуск VBA}

Чтобы получить доступ к макросам в текущей версии расчётного модуля для выполнения упражнений необходимо:
\begin{itemize}
	\item Запустить Excel запустив рабочую книгу для выполнения упражнений
	\item Нажать комбинацию клавиш <Alt-F11>
	\item Откроется новое окно c редактором макросов VBA (Рис. \ref{ris:VBA_overview}). Иногда в литературе окно редактирования макросов обозначают как VBE (Visual Basic Enviroment)
	\item Окне VBE можно изучить структуру проекта (набора макросов и других элементов). Раздел со структурой проекта можно открыть из меню <Вид – Обозреватель проекта>. Макросы располагаются в ветках «модули» и «модули классов»
	 
\end{itemize}

\begin{figure}[ht]
	\center{\includegraphics[width=1\linewidth]{VBA_overview}}
	\caption{Окно редактора VBE}
	\label{ris:VBA_overview}
\end{figure}

\section{Ключевые особенности VBA и соглашения, используемые в макросах}
Строки, начинающиеся со знака ‘ являются комментариями. В VBE они выделяются зелёным цветом. На исполнение макроса не влияют.

Для многих макросов не обязательно задавать все параметры. Некоторые значения параметров могут не задаваться – тогда будут использованы значения параметров, принятые по умолчанию. Параметры, допускающие задание по умолчанию помечены в исходном коде ключевым словом \mintinline{vb.net}{Optional}.

\section{Обозначение параметров}
При создании макросов в основном использовались международные обозначения переменных принятые в монографиях общества инженеров нефтяников SPE.
