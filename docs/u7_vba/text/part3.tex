\chapter{Функции модуля  «u7\_Excel\_functions\_ESP»}
В этом модули приведены интерфейсные функции Excel (функции, которые можно вызывать непосредственно с листа Excel) для расчёта параметров работы УЭЦН - установки электрического центробежного насоса. 

УЭЦН состоит из следующих основных конструктивных элементов:
\begin{itemize}
	\item ЦН - центробежный насос. Модуль обеспечивающий перекачку жидкости.
	\item ПЭД - погружной электрический двигатель. Модуль обеспечивающий преобразование электрической энергии, поступающий к УЭЦН по кабелю в механическую энергию вращения вала.
	\item ГС - газосепаратор или приемный модуль. Модуль обеспечивающий забор пластовой жидкости из скважины и подачу ее в насос. При этом центробежный газосепаратор способе отделить часть свободного газа в потоке и направить его в межтрубное пространство скважины.
	\item вал - узел передающий энергию от погружного электрического двигателя (ПЭД) к остальным узлам установки, в том числе к центробежному насосу.
\end{itemize}

Задача расчета УЭЦН обычно сводится к следующим:
\begin{itemize}
	\item Прямая задача - по заданным значения дебита жидкости скважины,  давлению на приеме, напряжению питания УЭЦН на поверхности найти давление на выкиде насоса, потребляему электрическую мощность, потребляемый ток установки, КПД всей системы и отдельных узлов системы
	\item Обратная задача - по данным контроля параметров работы УЭЦН на поверхности - потребляемому току, напряжению питания частоте подаваемого напряжения, данным по конструкции УЭЦН и скважины найти дебит жидкости и обводнённость по скважине, давление на приеме и забойное давление.
	\item Задача узлового анализа - по данным конструкции скважины, параметров работы погружного оборудования оценить дебит по жидкости скважины при заданным параметрах работы УЭЦН или при из изменении. К этому типу задач относится задача подбора погружного оборудования для достижения заданных условий эксплуатации 
	
\end{itemize}

Для расчёта УЭЦН требуется рассчитать гидравлические параметры работы ЦН и электромеханические параметры ПЭД

\section{Гидравлический расчет центробежного насоса (ЦН)}

Расчет выполняется на основе паспортных характеристик ЦН. 

\section{Электромеханический расчёт погружного электрического двигателя ПЭД}
Рассматривается асинхронный электрический двигатель. 

Погружные асинхронные электрические двигатели для добычи нефти выполяются трехфазными. 

Впервые конструкция трёхфазного асинхронного двигателя была разработана, создана и опробована русским инженером М. О. Доливо-Добровольским в 1889-91 годах. Демонстрация первых двигателей состоялась на Международной электротехнической выставке во Франкфурте на Майне в сентябре 1891 года. На выставке было представлено три трёхфазных двигателя разной мощности. Самый мощный из них имел мощность 1.5 кВт и использовался для приведения во вращение генератора постоянного тока. Конструкция асинхронного двигателя, предложенная Доливо-Добровольским, оказалась очень удачной и является основным видом конструкции этих двигателей до настоящего времени.

За прошедшие годы асинхронные двигатели нашли очень широкое применение в различных отраслях промышленности и сельского хозяйства. Их используют в электроприводе металлорежущих станков, подъёмно-транспортных машин, транспортёров, насосов, вентиляторов. Маломощные двигатели используются в устройствах автоматики.

Широкое применение асинхронных двигателей объясняется их достоинствами по сравнению с другими двигателями: высокая надёжность, возможность работы непосредственно от сети переменного тока, простота обслуживания.

Для расчёта электромеханических параметров погружных электрических двигателей полезно понимать теоретические основы их работы. Теория работы погружных асинхронных двигателей не отличаем от теории применимой к двигателям применяемым на поверхности. Далее кратко изложены основные положения теории. 

Трехфазная цепь является частным случаем многофазных систем электрических цепей, представляющих собой совокупность электрических цепей, в которых действуют синусоидальные ЭДС одинаковой частоты, отличающиеся по фазе одна от другой и создаваемые общим источником энергии.
Переменный ток протекающий по трехфазной цели характеризуется следующими параметрами:

\begin{itemize}
	\item Фазное напряжение $U_A, U_B, U_C $ - напряжение между линейным проводом и нейтралью
	\item Линейное напряжение $U_{AB}, U_{BC}, U_{CA} $ - напряжение между одноименными выводами разных фаз
	\item Фазный ток $I_{phase}$ – ток в фазах двигателя.
	\item Линейный ток $I_{line}$ – ток в линейных проводах.
	\item $ \cos \varphi $ - коэффициент мощности, где $ \varphi$ величина сдвига по фазе между напряжением и током 
\end{itemize}

Подключение двигателя к цепи трехфазного тока может быть выполнено по схеме "звезда" или "треугольник".

Тут нужен рисунок  

Для схемы звезда фазное напряжение меньше линейного в $\sqrt{3}$ раз.

$$ U_{AB} = \sqrt{3} U_{A} $$
$$ I_{phase} = I_{line} $$

Для схемы треугольник 

$$ U_{AB} =  U_{A} $$
$$ I_{line} =\sqrt{3} I_{phase} $$


В погружных двигателях обычно применяет схема подключения звезда. Эта схема обеспечивает более низкое напряжение в линии, что способствует повышению КПД передачи энергии по длинному кабелю. Еще есть причины?
При схеме подключения звезда токи в линии и в фазной обмотке статора двигателя совпадают, поэтому значение тока обозначают $I$ не указывая индекс в явном виде. Поскольку линейное напряжения проще измерить и легче контролировать параметры трехфазного двигателя обычно заданию линейный. в частности номинальное напряжение питания двигателя это линейное напряжение (напряжение между фазами). Далее линейное напряжение будет обозначать без индекса как $U$

Активная электрическая мощность в трехфазной цепи задается выражением 
$$ P= \sqrt{3}U I \cos \varphi$$

Реактивная мощность 
$$ Q= \sqrt{3}U I \sin \varphi$$

Соответственно полная мощность 
$$ S= \sqrt{3}U I $$

\subsection{ Устройство трёхфазной асинхронной машины}
Неподвижная часть машины называется статор, подвижная – ротор. Обмотка статора состоит из трёх отдельных частей, называемых фазами.

При подаче переменного напряжения и тока на обмотки статора внутри статора формируется вращающееся магнитное поле. Частота вращения магнитного поля совпадает с частотой питающего напряжения. 

Магнитный поток $\Phi $ и напряжение подаваемое на статор связаны приближенном соотношением 
$$ U_1 \approx E_1 = 4.44 w_1 k_1 f \Phi $$
где 

 $\Phi$ -  магнитный поток;
 
 $U_1$ -	напряжение в одной фазе статора;
 
 $f$   - частота сети;
 
 $E_1$	- ЭЦН в фазе статора;
 
 $w_1$ - число витков одной фазы обмотки статора;
 
 $k_1$  - обмоточный коэффициент.
   
Из этого выражения следует, что магнитный поток $\Phi $ в асинхронной машине не зависит от её режима работы, а при заданной частоте сети $f$ зависит только от действующего значения приложенного напряжения $U_1$


Для ЭДС ротора можно записать выражение 

$$  E_2 = 4.44 w_2 k_2 f S \Phi $$

где 


$S$ - величина скольжения (проскальзования);

$E_2$	- ЭЦН в фазе ротора;

$w_2$ - число витков одной фазы обмотки ротора;

$k_2$ - обмоточный коэффициент ротора.

ЭДС, наводимая в обмотке ротора, изменяется пропорционально скольжению и в режиме двигателя имеет наибольшее значение в момент пуска в ход.
Для тока ротора в общем случае можно получить такое соотношение

$$  I_2 = \frac{E_2 S}{\sqrt{R_2^2+(S X_2^2)}} $$

где 

$R_2$ -  активное сопротивление обмотки ротора, связанное с потерями на нагрев обмотки;  

$X_2 = 2 \pi f L_2$ - индуктивное сопротивление обмотки неподвижного ротора, связанное с потоком рассеяния;

Отсюда следует, что ток ротора зависит от скольжения и возрастает при его увеличении, но медленнее, чем ЭДС.

Для асинхронного двигателя можно получить следующее выражение для механического момента 

$$ M = \frac{1}{4.44 w_2 k_2 k_T^2 f} \frac{U_1^2 R_2 S}{R_2^2 + (S X_2^2)^2}$$

где 

$k_T = \frac{E_1}{E_2} = \frac{w_1 k_1}{w_2 k_2}$ - коэффициент трансформации асинхронной машины

Из полученного выражения для электромагнитного момента следует, что он сильно зависит от подведённого напряжения ($M \sim U_1^2$). При снижении, например, напряжения на 10\%, электромагнитный момент снизится на 19\% $M \sim (0,9U_1)^2=0.81 U_1^2)$. Это является одним из недостатков асинхронных двигателей. 




\newpage
