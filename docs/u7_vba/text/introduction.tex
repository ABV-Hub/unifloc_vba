\chapter*{Введение}                         % Заголовок
\addcontentsline{toc}{chapter}{Введение}    % Добавляем его в оглавление

Документ описывает расчётный модуль \unf реализованный в Excel VBA. Модуль предназначен для изучения математических моделей систем нефтедобычи и развития навыков проведения инженерных расчётов.

Расчётный модуль охватывают основные элементы математических моделей систем нефтедобычи - модель физико-химических свойств пластовых флюидов, модели многофазного потока в трубах, в пласте, задачи узлового анализа, модели скважинного оборудования в частности УЭЦН.  

Для использования \unf требуются навыки уверенного пользователя MS Excel, желательно знание основ программирования и основ теории добычи нефти. 

Алгоритмы реализованные в расчётном модуле не претендуют на полноту и достоверность и ориентированы на учебные задачи и проведение простых расчётов. Руководство пользователя также не претендует на полноту описания системы (часто получается, что описание отстаёт от текущего состояния \unf). Все приводится как есть. Более надёжным способом получения достоверной информации о работе макросов \unf является изучение непосредственно расчётного кода в редакторе VBE.

По всем вопросам можно обращаться к автору расчётных модулей - Хабибуллину Ринату Альфредовичу (khabibullin.ra@gubkin.ru)  

