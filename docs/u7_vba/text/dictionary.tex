\chapter*{Словарь терминов}             % Заголовок
\addcontentsline{toc}{chapter}{Словарь терминов}  % Добавляем его в оглавление

Словарь описывает термины и сокращения широко используемые в описании и в системе \unf.


\textbf{VBA} "--- Visual Basic for Application язык программрования встроенный в Excel и использованный для написания макросов \unf.

\textbf{VBE} "--- Среда разработки для языка VBA. Встроена в Excel.

\textbf{BHP, Pwf} "--- Bottom hole pressure. Well flowing pressure Забойное давление

\textbf{BHT, TBH} "--- Bottom hole temperature. Забойная температура

\textbf{WHP, PWH} "--- Well head pressure. Устьевое давление. Как правило, соответствует буферному давлению.

\textbf{WHT, TWH} "--- Well head temperature. Устьевая температура. Температура флюида на устье скважины. Температура в точке замера буферного давления.

\textbf{IPR} "--- Inflow performance relationship. Индикаторная кривая. Зависимость забойного давления от дебита для пласта. Широко используется в узловом анализе.

\textbf{VLP, VFP} "--- Vertical lift performance, vertical flow performance, outflow curve. Кривая лифта, кривая оттока. Зависимость забойного давления от дебита для скважины. Широко используется в узловом анализе.

\textbf{ZNLF} "--- Zero net liquid flow. Барботаж - движение газа через столб неподвижной жидкости. Соответствует условиям движения газа в затрубном пространстве при эксплуатации добывающих скважин с использованием погружных насосов.

\textbf{ЭЦН} "--- Электрический центробежный насос.

\textbf{УЭЦН} "--- Установка электрического центробежного насоса. Включает весь комплекс погружного и поверхностного оборудования необходимого для работы насоса - насос (ЭЦН), погружной электрический двигатель (ПЭД), гидрозащита (ГЗ), входной модуль (ВМ) и газосепаратор (ГС), электрический кабель, станция управления (СУ) и другие элементы

\textbf{ESP} "--- Electrical submersible pump. Электрический центробежный насос.

\textbf{GL} "--- Gas Lift. Газлифтный способ эксплуатации добывающих скважин. 

\textbf{РНХ ЭЦН} "--- Расходно напорная характеристика электрического центробежного насоса. Ключевая характеристика ЭЦН. Дается производителем в каталоге ЭЦН для новых насосов или определяется на стенде для ремонтных ЭЦН. 

\textbf{PVT} "--- Pressure Volume Temperature. Общепринятое обозначение для физико-химических свойств пластовых флюидов - нефти, газа и воды.

\textbf{MF} "--- MultiPhase. Много Фазный поток. Префикс для функций имеющих дело с расчетом многофазного потока в трубах и скважине.

\textbf{НКТ} "--- Насосно компрессорная труба. Часть конструкции скважины. по колонне НКТ добывается скважинная продукция или закачивается вода. Может быть заменена в процессе эксплуатации при ремонте скважины. 

\textbf{ЭК} "--- Эксплуатационная колонна. Часть конструкции скважины.  Не может быть заменена в процессе эксплуатации при ремонте скважины. 

\textbf{ГЖС} "--- Газо жидкостная смесь. Часто используется для обозначения совместно двигающихся флюидов в многофазном потоке - нефти, газа, воды.

