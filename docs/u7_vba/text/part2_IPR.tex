
\section{Расчет многофазного потока в пласте}
Для анализа работы скважины и скважинного оборудования в большинстве случаев достаточно простейшего подхода для описания производительности пласта. На текущий момент в \unf{} используется линейная индикаторная кривая с поправкой Вогеля для учета разгазирования в призабойной зоне пласта с учетом обводненности \cite{KBrown_AL_methods_vol4}. 

Пользовательские функции для расчета производительности пласта начинаются с префикса  \mintinline{vb.net}{IPR_}. 

Для расчета притока из пласта необходимо определить связь между дебитом жидкости $Q_{liq}$ (притоком) и забойным давлением работающей скважины $P_{wf}$.
Линейная индикаторная кривая на основе закона Дарси задает такую связь через коэффициент продуктивности скважины, который определяется как 
\begin{equation}\label{PI_def} 
 PI = \frac{Q_{liq}}{P_{res} - P_{wf}} 
\end{equation}

где $P_{res}$ - пластовое давление - давление на контуре питания скважины. Закон Дарси описывает установившийся приток несжимаемой жидкости в однородном пласте. 

Соответственно уравнение притока будет иметь вид

$$ Q_{liq} = PI \left( P_{res} - P_{wf} \right) $$

Для линейного притока по закону Дарси коэффициент продуктивности может быть оценен либо по данным эксплуатации из уравнения \ref{PI_def} либо по аналитической зависимости по характеристикам пласта и системы заканчивания. Например для радиального притока к вертикальной скважине широко известна формула Дюпюи согласно которой 
\begin{equation}\label{eq_Dupui} 
PI = f \cdot \frac{kh}{\mu B}\frac{1}{ \ln \dfrac{r_e}{r_w} + S }  
\end{equation}

здесь $f$ - размерный коэффициент, зависящий от выбранной системы единиц для остальных параметров. Так для системы единиц

\newcommand{\rnttab}[1]{
	\begin{tabular}[c]{@{}c@{}}#1\end{tabular}	
}

\begin{table}[]
	\centering
	\caption{Размерности параметров выражения \ref{eq_Dupui}} \label{tab:dim_Dupui}
	\begin{tabular}{|c|c|c|c|c|}
		\hline
		Обозначение & Параметр   			        	& СИ           & \rnttab{Практические \\ метрические}     & \rnttab{Американские\\ промысловые} \\ \hline
		$f$        & \rnttab{размерный \\ коэффициент} & $2\pi$       & $\dfrac{1}{18.41}$     			      & $\dfrac{1}{141.2}$                      \\ \hline
		$k$        & проницаемость           		    & $\text{м}^2$ & мД                    					  & mD   							    \\ \hline
		$h$        & \rnttab{мощность \\ пласта}       & м            & м                      				  & ft   								    \\ \hline		
		$B$        & \rnttab{объемный \\ коэффициент}  & $\text{м}^3/\text{м}^3$    & $\text{м}^3/\text{м}^3$    & $scf/bbl$    						\\ \hline
		$\mu$      & вязкость                           & Па $\cdot$ с & сП                                       & сP                                  \\ \hline
		$r_e$      & \rnttab{радиус зоны \\ дренирования} & м & м                                       & ft                                  \\ \hline
		$r_w$      & \rnttab{радиус  \\ скважины} & м & м                                       & ft                                  \\ \hline
		$S$        & скин фактор 				   & \multicolumn{3}{c|}{безразмерный}                     \\ \hline
	\end{tabular}
\end{table}
 
 При снижении забойного давления добывающей скважины ниже давления насыщения, оценка дебита жидкости по закону Дарси  оказывается завышенной. Газ выделяющийся в призабойной зоне пласта создает дополнительное гидравлическое сопротивление.  В \unf{} поправка на снижение забойного давления ниже давления насыщения реализована на основе поправки Вогеля. Для безводной нефти по Вогелю продуктивность скважины по данным тестовой эксплуатации - дебите жидкости $Q_{liq}$ и соответствующем забойном давлении $P_{wf}$ может быть оценен по выражению \ref{eq_Vogel}.
 
 \begin{equation}\label{eq_Vogel} 
 PI = \frac{Q_{liq}}{P_{res} - P_{b} + \dfrac{P_{b}}{1.8} \left[ 1.0 - 0.2  \dfrac{P_{wf}}{P_{b}}- 0.8 \left( \dfrac{P_{wf}}{P_{b}} \right)^2 \right] }   
 \end{equation}
 
 При наличии обводненности зависимость усложняется.
 
 В \unf{} реализована модель определения коэффициента продуктивности по данным эксплуатации. Сравнение индикаторных кривых, построенных по тестовым данным $Q_{liq} = 100$ и $P_{wf} = 150$ при наличии и отсутствии воды, приведено на рисунке. 
 
 \newcommand{\IPRFile}{data/IPR_fw_data.txt}
 \begin{tikzpicture}[scale=1]
 \begin{axis}[
 width=14cm,
 height=10cm,
 xlabel=$Q\; m^3 / day$,
 ylabel=$P_{wf} \; atma$,
 legend pos=north east,
 title=Pipe Pressure Drop]
 \addplot table [y=Pwf_fw0, x=Q_fw0]{\IPRFile};
 \addlegendentry{$f_w = 0$}
 \addplot table [y=Pwf_fw95, x=Q_fw95]{\IPRFile};
\addlegendentry{$f_w = 95$}
 \end{axis}
 \end{tikzpicture}
 
 

\subsection{IPR\_pi\_sm3dayatm – расчёт продуктивности}
Функция позволяет рассчитать коэффициент продуктивности скважины по данным тестовой эксплуатации. Особенность линейной модели притока к скважине с поправкой Волегя заключается в минимальном наборе исходных данных, необходимых для построения индикаторной кривой. Достаточно знать пластовое давление, дебит и забойное давление в одной точке.

\putlisting{listings/IPR_PI_sm3dayatm.lst}


\subsection{IPR\_pwf\_atm – расчёт забойного давления по дебиту и продуктивности}
Функция позволяет рассчитать забойное давление скважины по известным значениям дебита и продуктивности.

\putlisting{listings/IPR_Pwf_atma.lst}

\subsection{IPR\_qliq\_sm3day – расчёт дебита по забойному давлению и продуктивности}
Функция позволяет рассчитать дебит жидкости скважины на поверхности по забойному давлению и продуктивности.

\putlisting{listings/IPR_Qliq_sm3Day.lst}



\newpage
